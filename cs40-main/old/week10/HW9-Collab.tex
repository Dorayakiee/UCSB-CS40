\documentclass[12pt, oneside]{article}

\usepackage[letterpaper, scale=0.89, centering]{geometry}
\usepackage{amssymb,amsmath,pifont,amsfonts,comment,enumerate}
\usepackage{currfile,xstring,hyperref,tabularx,graphicx,wasysym}
\usepackage[labelformat=empty]{caption}
\usepackage[dvipsnames,table]{xcolor}
\usepackage{multicol,multirow,array,listings,tabularx,lastpage,textcomp,booktabs}



% NOTE: This environment is credit @pnpo (https://tex.stackexchange.com/a/218450)
\lstnewenvironment{algorithm}[1][] %defines the algorithm listing environment
{   
    \lstset{ %this is the stype
        mathescape=true,
        frame=tB,
        numbers=left, 
        numberstyle=\tiny,
        basicstyle=\rmfamily\scriptsize, 
        keywordstyle=\color{black}\bfseries,
        keywords={,procedure, div, for, to, input, output, return, datatype, function, in, if, else, foreach, while, begin, end, }
        numbers=left,
        xleftmargin=.04\textwidth,
        #1
    }
}
{}
\lstnewenvironment{java}[1][]
{   
    \lstset{
        language=java,
        mathescape=true,
        frame=tB,
        numbers=left, 
        numberstyle=\tiny,
        basicstyle=\ttfamily\scriptsize, 
        keywordstyle=\color{black}\bfseries,
        keywords={, int, double, for, return, if, else, while, }
        numbers=left,
        xleftmargin=.04\textwidth,
        #1
    }
}
{}

\newcommand\abs[1]{\lvert~#1~\rvert}
\newcommand{\st}{\mid}

\newcommand{\A}[0]{\texttt{A}}
\newcommand{\C}[0]{\texttt{C}}
\newcommand{\G}[0]{\texttt{G}}
\newcommand{\U}[0]{\texttt{U}}

\newcommand{\cmark}{\ding{51}}
\newcommand{\xmark}{\ding{55}}

\setlength{\parindent}{0em}
\setlength{\parskip}{1em}

\hypersetup{
    colorlinks=true,
    linkcolor=blue,
    filecolor=magenta,      
    urlcolor=cyan,
    pdftitle={Sharelatex Example},
    bookmarks=true,
    pdfpagemode=FullScreen,
}
\usepackage{enumitem}

\title{HW9 Collaborative}
\author{CS40 Fall'21\\\\\\
Benjamin Cruttenden (4672440)\\\
Bharat Kathi (5938444)\\\
Sean Oh (4824231)\\\
Marco Wong (4589198)}
\date{Due: Thursday, Dec 02, 2021 at 10:00PM on Gradescope}
\begin{document}
\maketitle

{\bf In this assignment,}

You will have more practice with proof strategies learned so far and using functions to compare the sizes of sets.


{\bf For all HW assignments:}

Please see the instructions and policies for assignments on the class website and on the writeup for HW1.  In particular, these policies address
\begin{multicols}{2}
\begin{itemize}
\item Collaboration policy
\item Where to get help
\item Typing your solutions
\item Expectations for full credit
\end{itemize}
\end{multicols}


You will submit this assignment via Gradescope
(\href{https://www.gradescope.com}{https://www.gradescope.com}) in the assignment called ``HW9-Collabrative''.

\newpage

{\bf Assigned questions}

\begin{enumerate}




\item Let $B = \{0, 1\}$. $B^n$ is the set of binary strings with $n$ bits. Define the set $E_n$ to be the set of binary strings with $n$ bits that have an even number of $1$'s. Note that zero is an even number, so a string with zero $1$'s (i.e., a string that is all $0$'s) has an even number of $1$'s.
\begin{enumerate}
    \item Find a bijection between $B^9$ and $E_{10}$ and prove that your function is a bijection.
    \begin{description}
        \item[Answer:] .\\
        $B9: 2^9 = 512$ different combinations\\
        $E10: (2^10)/2 = 1024/2 = 512$ different combinations\\
        $|B^9| = |E_10|$\\
        $|B^9| = 2^9 = 512$\\
        $|E_10| = 1024/2 = 512$\\
        $512 = 512$ so there is a bijection\\\\
        $000000000 \rightarrow 0000000000\\
        000000001 \rightarrow 0000000011\\
        000000010 \rightarrow 0000000101\\
        000000011 \rightarrow 0000000110$\\
        $f: E_10 -> B^9$\\
        $f(es) = e$, where $es$ in $E_{10}$ and $s$ in $\{0,1\}$\\\\
        Proof:\\\\
        f is well defined:\\
        Consider arbitrary $es$ in $E_{10}$. WTS it maps to exactly one element of $B^9$.\\
        Since $es$ in $E_{10}$, it is a binary string with 10 bits.\\
        Since $s$ in $\{0,1\}$ it is a binary string with 1 bit.\\
        Removing $s$ from $es$ gives a binary string with 9 bits, so e in $B^9$.\\
        Thus, each input for f maps to exactly 1 output.\\\\
        f is one-to-one:\\
        For $a,b$ in domain $E_{10}$, if $f(a) = f(b)$, then $a = b$\\
        We want to prove that if  $f(a) = f(b)$, then $a = b$\\
        Towards a direct proof, assume f(a) = f(b)\\
        Since $f(a) = f(b)$, $e_a = e_b$, by definition of $f$.\\
        Add arbitrary s in $\{0,1\}$ to both sides: $e_{as}$ = $e_{bs}$\\
        Then, a = b\\
    \end{description}
    \item What is $\vert E_{10}\vert$?
    \begin{description}
        \item[Answer:] .\\
        $|E_{10}| = (2^{10})/2 = 2^9 = 512$
    \end{description}
\end{enumerate}



\item Prove that $| \mathbb{N} | \leq |\mathcal{P} ( \mathbb{N} ) |$ by showing that the witness function $f:  \mathbb{N} \to\mathcal{P}(\mathbb{N}) $ with $f(x) = \{x^2\}$ is one-to-one.


\item Apply the pigeonhole principle/generalized pigeonhole principle to answer the following question. If the pigeonhole principle can not be applied, give a specific counterexample. 

\begin{enumerate}
    \item A team of four high jumpers all have a personal record that is at least $6$ feet and less than $7$ feet. Is it necessarily true that two of the team members must have personal records that are within (less than or equal to) four inches of each other?  Heights are measured to within a precision of $\frac{1}{4}$ inch. 
    \item There are 121.4 million people in the United States who earn an annual income that is at least \$$10,000$ and less than $\$1000,000$ dollars. Annual income is rounded to the nearest dollar. Show that there are 123 people who earn the same annual income in dollars.

\end{enumerate}



\item For each of the functions below, indicate whether the function is onto, one-to-one, neither or both. Justify your choice using an appropriate proof strategy

\begin{enumerate}
    \item $h: \mathbb{Z} \to \mathbb{Z}. h(x) = x^3$
    \item $g: \mathbb{Z} \times \mathbb{Z} \to \mathbb{Z} \times \mathbb{Z}. g(x, y) = (1-y, 1-x)$

\end{enumerate}

\item Consider the binary relation $R$ on the set of integers define as $R_m= \{(a,b) \in \mathbb{Z} \times \mathbb{Z} \mid a - b = 3\cdot m\}$ for some positive integer $m$.  Prove or disprove that $R_m$ is a equivalence relation. 






\item Extra credit: Recall the definitions of the greatest common divisor, gcd:
\label{gcd_proof}
{\bf Greatest common divisor} Let $a$ and $b$ be integers, not both zero. The largest integer $d$ such that 
$d$ is a  factor of $a$ and $d$ is a factor of  $b$ is called the greatest common divisor of $a$ and $b$ 
and is denoted by $gcd(a, b)$.


{\bf Lemma 2:} For every two integers $a$ and  $b$, not both zero, with  $gcd(a,b) = 1$, it is not the case that both $a$
is  even and $b$ is even.

{\bf Proof} of Lemma 2:

Towards a universal generalization, let $a$ and $b$ be integers, not both zero, with $gcd(a,b) = 1$. 
We will proceed in a {\bf proof by contradiction} to show that it is not the case that both $a$ is even and $b$ is even.

\ldots {\it Proof would continue here} \ldots


Since the goal in a proof by contradiction is to prove $\lnot p \to (r \land \lnot r)$ for some propositions $p$ and $r$, what would be {\bf assumed} in the 
next step of the proof?

Write this assumption both symbolically and in English. 

{\it You do not need to complete the proof for credit.}



\end{enumerate}



\end{document}