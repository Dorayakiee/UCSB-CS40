\documentclass[12pt, oneside]{article}



\documentclass{article}
\usepackage[letterpaper, scale=0.89, centering]{geometry}
\usepackage{amssymb,amsmath,pifont,amsfonts,comment,enumerate}
\usepackage{currfile,xstring,hyperref,tabularx,graphicx,wasysym}
\usepackage[labelformat=empty]{caption}
\usepackage[dvipsnames,table]{xcolor}
\usepackage{multicol,multirow,array,listings,tabularx,lastpage,textcomp,booktabs}



% NOTE: This environment is credit @pnpo (https://tex.stackexchange.com/a/218450)
\lstnewenvironment{algorithm}[1][] %defines the algorithm listing environment
{   
    \lstset{ %this is the stype
        mathescape=true,
        frame=tB,
        numbers=left, 
        numberstyle=\tiny,
        basicstyle=\rmfamily\scriptsize, 
        keywordstyle=\color{black}\bfseries,
        keywords={,procedure, div, for, to, input, output, return, datatype, function, in, if, else, foreach, while, begin, end, }
        numbers=left,
        xleftmargin=.04\textwidth,
        #1
    }
}
{}
\lstnewenvironment{java}[1][]
{   
    \lstset{
        language=java,
        mathescape=true,
        frame=tB,
        numbers=left, 
        numberstyle=\tiny,
        basicstyle=\ttfamily\scriptsize, 
        keywordstyle=\color{black}\bfseries,
        keywords={, int, double, for, return, if, else, while, }
        numbers=left,
        xleftmargin=.04\textwidth,
        #1
    }
}
{}

\newcommand\abs[1]{\lvert~#1~\rvert}
\newcommand{\st}{\mid}

\newcommand{\A}[0]{\texttt{A}}
\newcommand{\C}[0]{\texttt{C}}
\newcommand{\G}[0]{\texttt{G}}
\newcommand{\U}[0]{\texttt{U}}

\newcommand{\cmark}{\ding{51}}
\newcommand{\xmark}{\ding{55}}

\setlength{\parindent}{0em}
\setlength{\parskip}{1em}

\hypersetup{
    colorlinks=true,
    linkcolor=blue,
    filecolor=magenta,      
    urlcolor=cyan,
    pdftitle={Sharelatex Example},
    bookmarks=true,
    pdfpagemode=FullScreen,
}
\usepackage{graphicx} % Required for inserting images

\title{HW1}
\author{Zixuan Chen}
\date{June 2024}

\begin{document}

\maketitle

\section{Introduction}

\end{document}


\usepackage{enumitem}

\title{HW9 Individual}
\author{CS40 Fall'21\\\\\\
Bharat Kathi (5938444)}
\date{Due: Monday, Dec 6, 2021 at 10:00PM on Gradescope}
\begin{document}
\maketitle

{\bf In this assignment,}

You will have more practice with counting, binary relations, and proof strategies learned so far.


{\bf For all HW assignments:}

Please see the instructions and policies for assignments on the class website and on the writeup for HW1.  In particular, these policies address
\begin{multicols}{2}
\begin{itemize}
\item Collaboration policy
\item Where to get help
\item Typing your solutions
\item Expectations for full credit
\end{itemize}
\end{multicols}


You will submit this assignment via Gradescope
(\href{https://www.gradescope.com}{https://www.gradescope.com}) in the assignment called ``HW9-Individual''.

\newpage

{\bf Assigned questions}

\begin{enumerate}


\item We define the following function: $$f: \{0,1\}^3 \rightarrow \{0,1\}^3,$$ where the output of $f$ is obtained by taking the input string and replacing the first bit by $1$, regardless of whether the first bit is a $0$ or $1$. For example, $f(001) = 101$ and $f(110) = 110$. Indicate whether the $f$ is onto, one-to-one, neither or both. If the function is not onto or not one-to-one, give an example showing why. 

\begin{description}
    \item[Answer:] .\\
        $\{0,1\}^3 = \{000,001,011,100,101,010,110,111\}$\\
        $f(000) = 100\\
        f(001) = 101\\
        f(011) = 111\\
        f(101) = 101\\
        f(010) = 110\\
        f(111) = 111\\
        f(110) = 110$\\\\
        Since f(000) and f (100) both equal 100, the function is not one-to-one. Since there is no preimage for 000, the function is also not onto.
\end{description}

\item Count the number of different one-to-one functions $f: \{0,1\}^7 \rightarrow \{0,1\}^7$. Justify your answer.

\begin{description}
    \item[Answer:] .\\
        Since $\{0,1\}^7$ is the set of all 7 long binary strings, the total number of elements is $2^7 = 128$.\\
        Since $f: \{0,1\}^7 \rightarrow \{0,1\}^7$ should be a one-to-one function, each element in $\{0,1\}^7$ must be mapped to a unique element in $\{0,1\}^7$.\\
        That means that we now have $2^7$ choices for the first term, $2^7-1$ choices for the second term, and so on.\\
        Therefore, the total number of choices is $128!*(2^7)!$
\end{description}

\item Recall that 
in a movie recommendation system, each 
user's ratings of movies is represented as a $n$-tuple (with the positive integer $n$  being the number of movies in the database), and each  component of 
the $n$-tuple is an element of  the collection $\{-1,0,1\}$.

Assume there are five movies in the database, so that   each user's ratings
can be represented as a $5$-tuple. Let $R$ be the set of all ratings, that  is, the  set of all $5$-tuples where each  component of the $5$-tuple is an element of  the collection $\{-1,0,1\}$.

Consider the following two binary relations on $R$:
\[
A_1 =  \{  (u,v) \in R \times R \mid 
\text{users $u$  and  $v$  agree about the first 
movie in the database} \}
\]
\[
A =  \{  (u,v) \in R \times R \mid 
\text{users $u$  and  $v$  don't care or haven't seen the same number of movies} \}
\]
Binary  relations that satisfy certain 
properties (namely,  are  reflexive, symmetric,
and transitive)  can help us group 
elements in a set into categories. 


\begin{enumerate}
    \item {\bf True} or {\bf False}: 
    The  relation $A_1$ holds of  $u=(1,1,1,1,1)$ and
    $v=(-1,-1,-1,-1,-1)$.
    \begin{description}
        \item[Answer:] True
    \end{description}
    \item {\bf True} or {\bf False}: 
    The  relation $A$ holds of  $u=(1,0,1,0,-1)$ and
    $v=(-1,0,1,-1,-1)$.
    \item {\bf True} or {\bf False}: $A_1$ is reflexive; namely, 
    $\forall u  \in R ~(~(u,u) \in A_1~)$
    \item {\bf True} or {\bf False}:  $A_1$ is symmetric; namely, 
    $\forall u \in  R~\forall  v \in R ~(~(u,v) \in A_1 \to  (v,u) \in A_1~)$
    \item {\bf True} or {\bf False}:  $A_1$ is transitive; namely, 
    $\forall u \in  R~\forall  v \in R ~\forall w \in  R (~\left( (u,v) \in A_1 \wedge (v,w)\in A_1\right) \to  (u,w) \in A_1~)$
    \item {\bf True} or {\bf False}:  $A$ is reflexive; namely, 
    $\forall u  \in R ~(~(u,u) \in A~)$
    \item {\bf True} or {\bf False}:  $A$ is anti-symmetric; namely, 
    $\forall u \in  R~\forall  v \in R ~(~(~(u,v) \in A \land  (v,u) \in A~) \to (u = v)~)$
    \item {\bf True} or {\bf False}:  $A$ is transitive; namely, 
    $\forall u \in  R~\forall  v \in R ~\forall w \in  R (~\left( (u,v) \in A \wedge (v,w)\in A\right) \to  (u,w) \in A~)$
\end{enumerate}

\item In the previous question select any one of parts (c) to (h) that evaluated to True and provide a formal proof using the strategies you have learned in CS40

\item {\it No justifications are required for credit for this question. It's a good idea to think about how you would explain how you arrived at your examples.} Given the relations $A_1$ and $A$ in Q4 answer the following questions:
\begin{enumerate}
    \item Give two distinct examples of elements in $[~(1,0,0,0,0)~]_{A_1}$
    \begin{description}
        \item[Answer:] (1,1,0,0,0) (1,-1,0,0,0)
    \end{description}
    \item Give two distinct examples of elements in $[~(1,0,0,0,0)~]_{A}$
    \begin{description}
        \item[Answer:] (0,1,0,0,0) (0,0,1,0,0)
    \end{description}
    \item Find examples $u, v \in R$ where $[ u]_{A_1} \neq [v]_{A_1}$ but $[ u]_{A} = [v]_{A}$ 
    \begin{description}
        \item[Answer:] $u = (1,0,0,0,0)$ and $v = (-1,0,0,0,0)$
    \end{description}
    \item Find examples $u, v \in R$ (different from the previous part) where $[ u]_{A_1} = [v]_{A_1}$ but $[ u]_{A} \neq [v]_{A}$ 
    \begin{description}
        \item[Answer:] $u = (1,0,0,0,1)$ and $v = (1,0,0,1,1)$
    \end{description}
\end{enumerate}

\item {\bf Bonus - not for credit (but much appreciated):} Please complete the course ESCI and TA evaluations by Dec 3 (Friday). 

\end{enumerate}



\end{document}