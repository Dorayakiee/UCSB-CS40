\documentclass[12pt, oneside]{article}



\documentclass{article}
\usepackage[letterpaper, scale=0.89, centering]{geometry}
\usepackage{amssymb,amsmath,pifont,amsfonts,comment,enumerate}
\usepackage{currfile,xstring,hyperref,tabularx,graphicx,wasysym}
\usepackage[labelformat=empty]{caption}
\usepackage[dvipsnames,table]{xcolor}
\usepackage{multicol,multirow,array,listings,tabularx,lastpage,textcomp,booktabs}



% NOTE: This environment is credit @pnpo (https://tex.stackexchange.com/a/218450)
\lstnewenvironment{algorithm}[1][] %defines the algorithm listing environment
{   
    \lstset{ %this is the stype
        mathescape=true,
        frame=tB,
        numbers=left, 
        numberstyle=\tiny,
        basicstyle=\rmfamily\scriptsize, 
        keywordstyle=\color{black}\bfseries,
        keywords={,procedure, div, for, to, input, output, return, datatype, function, in, if, else, foreach, while, begin, end, }
        numbers=left,
        xleftmargin=.04\textwidth,
        #1
    }
}
{}
\lstnewenvironment{java}[1][]
{   
    \lstset{
        language=java,
        mathescape=true,
        frame=tB,
        numbers=left, 
        numberstyle=\tiny,
        basicstyle=\ttfamily\scriptsize, 
        keywordstyle=\color{black}\bfseries,
        keywords={, int, double, for, return, if, else, while, }
        numbers=left,
        xleftmargin=.04\textwidth,
        #1
    }
}
{}

\newcommand\abs[1]{\lvert~#1~\rvert}
\newcommand{\st}{\mid}

\newcommand{\A}[0]{\texttt{A}}
\newcommand{\C}[0]{\texttt{C}}
\newcommand{\G}[0]{\texttt{G}}
\newcommand{\U}[0]{\texttt{U}}

\newcommand{\cmark}{\ding{51}}
\newcommand{\xmark}{\ding{55}}

\setlength{\parindent}{0em}
\setlength{\parskip}{1em}

\hypersetup{
    colorlinks=true,
    linkcolor=blue,
    filecolor=magenta,      
    urlcolor=cyan,
    pdftitle={Sharelatex Example},
    bookmarks=true,
    pdfpagemode=FullScreen,
}
\usepackage{graphicx} % Required for inserting images

\title{HW1}
\author{Zixuan Chen}
\date{June 2024}

\begin{document}

\maketitle

\section{Introduction}

\end{document}


\usepackage{enumitem}

\title{HW5 Collaborative}
\author{CS40 Fall'21\\\\\\
Benjamin Cruttenden (4672440)\\\
Bharat Kathi (5938444)\\\
Sean Oh (4824231)\\\
Marco Wong (4589198)}
\date{Due: Thursday, Nov 4, 2021 at 10:00PM on Gradescope}
\begin{document}
\maketitle

{\bf For all Collaborative HW assignments:}

Collaborative homeworks may be done individually or in groups of up to 4 students. You may switch HW partners for different HW assignments. The lowest HW score will not be included in your overall HW average. Please ensure your name(s) and PID(s) are clearly visible on the first page of your homework submission.

All submitted homework for this class must be typed. Diagrams may be hand-drawn and scanned and included in the typed document. You can use a word processing editor if you like (Microsoft Word, Open Office, Notepad, Vim, Google Docs, etc.) but you might find it useful to take this opportunity to learn LaTeX. LaTeX is a markup language used widely in computer science and mathematics. The homework assignments are typed using LaTeX and you can use the source files as templates for typesetting your solutions.\footnote{To use this template, you will need to copy both the source file (extension \texttt{.tex})  you'll be editing
and the file containing all the ``shortcut" commands we've defined for this class \href{https://drive.google.com/file/d/1FmQvgByKnNjTpIkAUw31TGWYrQZM-HK0/view?usp=sharing})}


{\bf Integrity reminders}
\begin{itemize}
\item Problems should be solved together, not divided up between the partners. The homework is
designed to give you practice with the main concepts and techniques of the course, while getting to know and learn from your classmates.
\item You may not collaborate on homework with anyone other than your group members.
You may ask questions about the homework in office hours (of the instructor, TAs, and/or tutors) and 
on Piazza.  You \emph{cannot} use any online resources about the course content other than the text
book and class material from this quarter -- this is primarily to ensure that we all use consistent notation and
definitions we will use this quarter.
\item Do not share written solutions or partial solutions for homework with other students in the class who are not in your group. Doing so would dilute their learning experience and detract from their success in the class.
\end{itemize}


You will submit this assignment via Gradescope
(\href{https://www.gradescope.com}{https://www.gradescope.com}) in the assignment called ``HW5-Collaborative''.


\newpage
\section*{Assigned Questions}
\begin{enumerate}
\item ({\it Graded for correctness}\footnote{This means your solution will be evaluated on the correctness only. No justification is needed.}) Colors can be described as amounts of red, green, and blue mixed together\footnote{This RGB representation
is common in web applications.  Many online tools are available to play around with mixing these colors, e.g. \url{https://www.w3schools.com/colors/colors_rgb.asp}}.  Mathematically, a color can be represented as a $3$-tuple $(r, g, b)$ where $r$
represents the red component, $g$ the green component, $b$ the blue component and where each of $r$, $g$, $b$ must
be a value from this collection of numbers from $0$ to $255$:
\label{sets_colors}
\begin{quote}
$\{$0, 1, 2, \ldots, 255$\}$
\end{quote}

\begin{enumerate}
\item \textbf{True} or \textbf{False}: $(1, 3, 4)$ fits the definition of a color above.
\begin{description}
  \item[Answer:] True
\end{description}
\item \textbf{True} or \textbf{False}: $(1, 100, 200, 0)$ fits the definition of a color above.
\begin{description}
  \item[Answer:] False
\end{description}
\item \textbf{True} or \textbf{False}: $(510, 255)$ fits the definition of a color above.
\begin{description}
  \item[Answer:] False
\end{description}
\item \textbf{True} or \textbf{False}: There is a color $(r_1, g_1, b_1)$ where $r_1 + g_1 + b_1$ is greater than $765$.
\begin{description}
  \item[Answer:] False
\end{description}
\item \textbf{True} or \textbf{False}: There is a color $(r_2, g_2, b_2)$ where $r_2 + g_2 + b_2$ is equal to $1$.
\begin{description}
  \item[Answer:] True
\end{description}
\item \textbf{True} or \textbf{False}: Another way to write the collection of allowed values for red, green, and blue components is $$\{x \in \mathbb{N}\mid 0 \leq x \leq 255 \}$$
\begin{description}
  \item[Answer:] False
\end{description}
\item \textbf{True} or \textbf{False}: Another way to write the collection of allowed values for red, green, and blue components is $$\{n \in \mathbb{Z}\mid 0 \leq n \leq 255 \}$$
\begin{description}
  \item[Answer:] True
\end{description}
\item \textbf{True} or \textbf{False}: Another way to write the collection of allowed values for red, green, and blue components is $$\{y \in \mathbb{Z}\mid -1 < y \leq 255 \}$$
\begin{description}
  \item[Answer:] True
\end{description}
\end{enumerate}

\item ({\it Graded for correctness}) For many applications in cryptography and random number generation,
dividing very large integers efficiently is critical.  Recall {\bf The Division Algorithm} (zyBook 5.2):
 Let $n$ be an integer 
and $d$ a positive integer. There are unique integers $q$ and $r$, with $0 \leq r < d$, such that 
$n = dq + r$. In this case, $d$ is called the divisor, $n$ is called the dividend, $q$ is called the quotient, 
and $r$ is called the remainder. We write $q=n \textbf{ div } d$ and $r=n \textbf{ mod } d$.

One application of the Division Algorithm is in computing the integer part of the logarithm.
When we discuss algorithms in this class, we will usually write them in 
pseudocode or English. Sometimes we will find it useful to relate the pseudocode to
runnable code in a programming language. We will typically use C++ for this.

\label{algo_division}

\begin{multicols}{2}
\begin{algorithm}[caption={Calculating log in pseudocode}]
procedure $\textit{log}$($n$: a positive integer)
$r$ := $0$
while $n$ > $1$
  $r$ := $r + 1$
  $n$ := $n$ div $2$
return $r$ $\{ r~\textrm{holds the result of the}~\log~\textrm{operation}\} $
\end{algorithm}
\columnbreak
\begin{java}[caption={Calculating log in C++}]
int log(int n) {
  if (n < 1) { 
    cerr<<``Illegal ArgumentException''<<endl; 
  }
  int result = 0;
  while(n > 1) {
    result = result + 1;
    n = n / 2;
  }
  return result;
}
\end{java}
\end{multicols}


\begin{enumerate}
\item Calculate $2027 \textbf{ div } 20$.  {\it You may use a calculator if you like.}
\begin{description}
  \item[Answer:] 101
\end{description}
\item Calculate $2027 \textbf{ mod } 20$.  {\it You may use a calculator if you like.}
\begin{description}
  \item[Answer:] 7
\end{description}
\item Calculate $(-347) \textbf{ div } 5$ {\it You may use a calculator if you like.}
\begin{description}
  \item[Answer:] -69
\end{description}
\item Calculate $(-347) \textbf{ mod } 5$ {\it You may use a calculator if you like.}
\begin{description}
  \item[Answer:] 3
\end{description}
\item How many different possible values of $r$ (results of taking $n \textbf{ mod } d$) are there when 
$n$ is any positive integer and $d$ is $20$?
\begin{description}
  \item[Answer:] 20 different values $0 <= r < 20$
\end{description}
\item What is the smallest positive integer $n$ which can be written as $16q+9$ for $q$ an integer?
\begin{description}
  \item[Answer:] 9, when q = 0
\end{description}
\end{enumerate}

\item Prove or disprove each of the following statements.
\begin{enumerate}
    \item For all integers $a$, $b$, and $c$, if $a \mid b$ and $a \mid c$, then $a \mid (b + c)$
    9, when q = 0
    \begin{description}
      \item[Answer:] By direct proof. Assuming a, b, and c are all integers. Also, a divides b so $b = ka$ for some integer k and since a divides c, then $c = ja$ for some integer j. Plugging these into the expression  $b + c = (ka) + (ja) = a(k+j)$. Since $(k+j)$ is an integer multiple of a, then $a|(b+c)$.
    \end{description}
    \item For all integers $a$, $b$, and $c$, if $a \mid b$ and $a \mid c$, then $a \mid bc$
    \begin{description}
      \item[Answer:] Towards a direct proof, assume $a | b$ and $a | c$ to show that $a | bc$, or that $bc = ka$ for some integer k. By definition,$b = ia$ for some integer i and $c = ja$ for some integer j. Plugging in b and c into bc, we get: $bc = (ia)(ja) = (ija)a = ka$. Since i,j, and a are all integers, $k = (ija)$ is also an integer. Since bc can be expressed as an integer times a, $a | bc$.      
    \end{description}
    \item For all integers $a$, $b$, and $c$, if $a \mid b$ or $a \mid c$, then $a \mid bc$
    \begin{description}
      \item[Answer:] Let a,b,c be arbitrary integers. Towards a proof by cases. Let case 1 be $(a | b) \land \lnot (a | c)$. Let case 2 be $\lnot (a | b) \land (a | c)$. We want to show for all cases that $a | bc$.\\\\
      Case 1:\\
      Assume that $a|b$, to show $a|bc$. Since $a|b$, $b = ka$ for some integer $k$. Plugging in $b$ to $bc$:\\ $bc = (ka)c = (kc)a$\\
      Since k and c are integers, $kc$ is also an integer. Since bc can be expressed as an integer times a, $a|bc$.\\\\
      Case 2:\\
      Assume that $a|c$, to show $a|bc$. Since $a|c$, $c = ja$ for some integer j. Plugging in c to $bc$:\\$bc = b(ja) = (jb)a$\\
      Since j and b are integers, $jb$ is also an integer. Since $bc$ can be expressed as an integer times a, $a|bc$.
    \end{description}
  \end{enumerate}




\item For each of the following inputs: (a) Use the template of the table provided above to trace Euclid's algorithm, (b) find the gcd of the two numbers by writing the output of the algorithm, (c)  express the gcd as a linear combination of the two inputs. 

\begin{algorithm}[caption={Euclidean algorithm}]
procedure $\textit{gcd}$($x,y$: positive integers)
$a$ := $x$
$b$ := $y$
if $a>b$
  swap $a$ and $b$
while $a\neq 0$
  $r$ := $b$ mod $a$
  $b$ := $a$
  $a$ := $r$
return $b$ $\{gcd(x,y)=b\} $
\end{algorithm}

{\bf Template of table to trace Euclid's algorithm}\\ \\\
\begin{minipage}{3.2in}
\begin{tabular}{c|c|c|c|c|c|c|}
$x$ & $y$  & $r$ & $a$ & $b$ & $a \neq 0$?\\
\hline 
 &  &  &  &  &\\
 &  &  &  &  &\\
 &  &  &  &  &\\
 &  &  &  &  &\\
\end{tabular}

$gcd( x , y) = ?$
\end{minipage}

\begin{enumerate}
    \item[(i)] $\gcd(81, 65)$
    \begin{description}
      \item[Answer:] .\\
      \begin{tabular}{c|c|c|c|c|c|c|}
        $x$ & $y$  & $r$ & $a$ & $b$ & $a \neq 0$?\\
        \hline 
         81 & 65 & 16 & 16 & 65 & T\\
         81 & 65 & 1 & 1 & 16 & T\\
         81 & 65 & 0 & 0 & 1 & F\\
        \end{tabular}\\\\
        (b) 1. b was returned, and looking at the trace of the algorithm b ended as 1.\\
        (c) 1 = -4*81 +5*65
    \end{description}
    \item[(ii)] $\gcd(279, 77)$
    \begin{description}
      \item[Answer:] .\\
      \begin{tabular}{c|c|c|c|c|c|c|}
        $x$ & $y$  & $r$ & $a$ & $b$ & $a \neq 0$?\\
        \hline 
        279 & 77 & 48 & 48 & 77 & T\\
        279 & 77 & 29 & 29 & 48 & T\\
        279 & 77 & 19 & 19 & 29 & T\\
        279 & 77 & 10 & 10 & 19 & T\\
        279 & 77 & 9 & 9 & 10 & T\\
        279 & 77 & 1 & 1 & 9 & T\\
        279 & 77 & 0 & 0 & 1 & T\\
        \end{tabular}\\\\
        (b) 1. b was returned, and looking at the trace of the algorithm b ended as 1.\\
        (c) 1 = -8*279 + 29*77
    \end{description}
\end{enumerate}



\item For each $x$ and $n$, find the multiplicative inverse $\textbf{ mod } n$ of $x$. Your answer should be an integer $s$ in the range $0$ through $n - 1$. Check your solution by verifying that $sx \textbf{ mod } n = 1$.
\begin{enumerate}
    \item $x = 35, n = 48$
    \begin{description}
      \item[Answer:] .\\
      x = 35 , n = 48\\
      gcd(48,35)\\
      Euclidean Sequence of gcd(48,35) yields:\\
      48	35	13	9	4	1	0\\
      so gcd(48,35) is 1, we can find multiplicative inverse mod 48 of 35 then by: \\

      Linear combination: 1 = 11*35 - 8*48\\

      11 * 35 mod 48 = 1\\
      Multiplicative inverse mod is 11\\

      11*35 mod 48 = 1
    \end{description}
    \item $x = 34, n = 55$
    \begin{description}
      \item[Answer:] .\\
      x= 34, n = 55\\
      The Euclidean algorithm for gcd(34,55) in the left column:\\\\
      \begin{tabular}{c|c}
        55 = 1*34 + 21 & 21 = 55 - 1*34\\
        34 = 1* 21 + 13 & 13 = 34 - 1*21\\
        21 = 1*13 +8 & 8 = 21 - 1*13\\
        13 = 1*8 + 5 & 5 = 13 - 1*8 \\
        8 = 1*5 + 3 & 3 = 8 - 1*5 \\
        5 = 1*3 + 2 & 2 = 5 - 1*3 \\
        3 = 1*2 + 1 & 1 = 3 - 1*2 \\
        2 = 2*1 + 0 &
      \end{tabular}\\\\
      gcd(34,55) = 1\\
      Using the right column, plug in values into 1 = 3 - 1*2 to get:\\
      Linear Combination: 1 = 34*34 - 21*55\\
      The inverse mod n of x is 34\\
      Check: (34)(34) mod (55) = 1\\
      Thus, s = 34
    \end{description}
\end{enumerate}


\quad



\end{enumerate}

\end{document}