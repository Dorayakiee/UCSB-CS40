\documentclass[12pt, oneside]{article}



\documentclass{article}
\usepackage[letterpaper, scale=0.89, centering]{geometry}
\usepackage{amssymb,amsmath,pifont,amsfonts,comment,enumerate}
\usepackage{currfile,xstring,hyperref,tabularx,graphicx,wasysym}
\usepackage[labelformat=empty]{caption}
\usepackage[dvipsnames,table]{xcolor}
\usepackage{multicol,multirow,array,listings,tabularx,lastpage,textcomp,booktabs}



% NOTE: This environment is credit @pnpo (https://tex.stackexchange.com/a/218450)
\lstnewenvironment{algorithm}[1][] %defines the algorithm listing environment
{   
    \lstset{ %this is the stype
        mathescape=true,
        frame=tB,
        numbers=left, 
        numberstyle=\tiny,
        basicstyle=\rmfamily\scriptsize, 
        keywordstyle=\color{black}\bfseries,
        keywords={,procedure, div, for, to, input, output, return, datatype, function, in, if, else, foreach, while, begin, end, }
        numbers=left,
        xleftmargin=.04\textwidth,
        #1
    }
}
{}
\lstnewenvironment{java}[1][]
{   
    \lstset{
        language=java,
        mathescape=true,
        frame=tB,
        numbers=left, 
        numberstyle=\tiny,
        basicstyle=\ttfamily\scriptsize, 
        keywordstyle=\color{black}\bfseries,
        keywords={, int, double, for, return, if, else, while, }
        numbers=left,
        xleftmargin=.04\textwidth,
        #1
    }
}
{}

\newcommand\abs[1]{\lvert~#1~\rvert}
\newcommand{\st}{\mid}

\newcommand{\A}[0]{\texttt{A}}
\newcommand{\C}[0]{\texttt{C}}
\newcommand{\G}[0]{\texttt{G}}
\newcommand{\U}[0]{\texttt{U}}

\newcommand{\cmark}{\ding{51}}
\newcommand{\xmark}{\ding{55}}

\setlength{\parindent}{0em}
\setlength{\parskip}{1em}

\hypersetup{
    colorlinks=true,
    linkcolor=blue,
    filecolor=magenta,      
    urlcolor=cyan,
    pdftitle={Sharelatex Example},
    bookmarks=true,
    pdfpagemode=FullScreen,
}
\usepackage{graphicx} % Required for inserting images

\title{HW1}
\author{Zixuan Chen}
\date{June 2024}

\begin{document}

\maketitle

\section{Introduction}

\end{document}


\usepackage{enumitem}

\title{HW4 Individual}
\author{CS40 Fall'21\\\\\\
Bharat Kathi (5938444)}
\date{Due: Monday, Nov 1, 2021 at 10:00PM on Gradescope}
\begin{document}
\maketitle

{\bf Integrity reminders for individual homeworks}
\begin{itemize}
\item ``Individual homeworks'' must be solely your own work. 
\item You may not collaborate on individual homeworks with anyone or seek help from online tutors or entities outside the class.
\item You may ask questions about the homework in office hours (of the instructor, TAs, and/or tutors) and 
on Piazza.  However, the staff will only answer clarifying questions on these homeworks. You \emph{cannot} use any online resources about the course content other than the text
book and class material from this quarter.
\item Do not share written solutions or partial solutions for homework with other students. Doing so would dilute their learning experience and detract from their success in the class.
\end{itemize}

You will submit this assignment via Gradescope
(\href{https://www.gradescope.com}{https://www.gradescope.com}) in the assignment called ``HW4-Individual''.

\subsection*{Summary of Proof Strategies (so far)}
In your proofs and disproofs of statements below, justify each  step
by reference to  a component of the  following proof  strategies
we  have discussed so far, and/or to relevant definitions and calculations.
\begin{itemize}
    \item A counterexample can be used to prove that  $\forall x P(x)$ is {\bf false}.
    \item  A witness can be used  to  prove that  $\exists x P(x)$ is {\bf true}.
    \item {\bf Proof of universal by exhaustion}: To prove that $\forall x \, P(x)$
is true when $P$ has a finite domain, evaluate the predicate at {\bf each} domain element to confirm that it is always T.
    \item  {\bf Proof by universal generalization}: To prove that $\forall x \, P(x)$
is true, we can take an arbitrary element $e$ from the domain and show that $P(e)$ is true, without making any assumptions about $e$ other than that it comes from the domain.
    \item To  prove  that $\exists x P(x)$ is {\bf false}, write the universal statement that is logically equivalent to its negation and then prove it true using universal generalization.
    \item {\bf Strategies for conjunction}: To prove that $p \land q$ is true, have two subgoals: subgoal (1) prove $p$ 
is  true; and, subgoal (2) prove $q$ is true. To prove that $p \land q$ is false, it's enough to prove that $p$ is false.
 To prove that $p \land q$ is false, it's enough to prove that $q$ is false.
    \item {\bf Proof of Conditional by Direct Proof}: To prove that the implication $p \to q$ is true, we can assume $p$ is true and use that assumption to show $q$ is true.
    \item {\bf Proof of Conditional by Contrapositive Proof}: To prove that the implication $p \to q$ is true, we can assume $\neg q$ is true and use that assumption to show $\neg p$ is true.
   
\end{itemize}


\newpage
\section*{Assigned Questions}
\begin{enumerate}

\item \textbf{Theorem}: If $n$ and $m$ are odd integers, then $n^2+m^2$ is even. 

For each of the following proof attempts of the theorem, explain where the proof uses invalid reasoning or skips essential steps.
\begin{enumerate}
    \item Let $n$ and $m$ be odd integers. Then $n=2k+1$ and $m=2j+1$. Plugging into the expression $n^2+m^2$ gives
    \[n^2+m^2=(2k+1)^2+(2j+1)^2=4k^2+4k+1+4j^2+4j+1=2(2k^2+2k+2j^2+2j+1)\]
    Since $k$ and $j$ are integers, $2k^2+2k+2j^2+2j+1$ is also an integer. Then $n^2+m^2$ equals to two times an integer, therefore $n^2+m^2$ is even.
    
    \begin{description}
        \item[Answer:].\\
        The definitions of k and j in the first part are missing.\\
        $$ n = 2k + 1, k \in Z $$
        $$ m = 2j + 1, j \in Z $$
        The rest of the proof is correct, just needs to include that since $k \in Z$ and $j \in Z$, then $n^2+m^2$ will be equal to 2 times an integer.
    \end{description}

    \item Let $n$ and $m$ be odd integers. Since $n$ is an odd integer, then $n = 2k+1$ for some integer $k$. Since $m$ is an odd integer, then $m = 2j+1$ for some integer $j$. Plugging in $2k+1$ for $n$ and $2j+1$ for $m$ into the expression $n^2 + m^2$ gives
    \[n^2+m^2=(2k+1)^2+(2j+1)^2\]
    Since $n^2 + m^2$ is equal to two times an integer, then $n^2 + m^2$ is an even integer.
    
    \begin{description}
        \item[Answer:].\\
        This argument has an invalid conclusion. $n^2 + m^2$ is not equal to 2 times an integer, it is equal to the sum of the squares of two integers.
    \end{description}
    
    \item Let $n$ and $m$ be odd integers. Since $n$ is an odd integer, then $n = 2k+1$ for some integer $k$. Since $m$ is an odd integer, then $m = 2k+1$ for some integer $k$. Plugging into the expression $n^2 + m^2$ gives
    \[n^2+m^2=(2k+1)^2+(2k+1)^2=2(4k^2+4k+1)\]
    Since $k$ is an integer, $4k^2+4k+1$ is also an integer. Since $n^2 + m^2$ is equal to two times an integer, then $n^2 + m^2$ is an even integer.

    \begin{description}
        \item[Answer:].\\
        The beginning of this argument is invalid. It is assuming that $m=n$ as both $m$ and $n$ are being set to $2k+1$, which is not something that can be assumed. We have to assume that m and n are distinct for the argument to work. The conclusion given only works when $m=n$.
    \end{description}

\end{enumerate}

\quad
\item \textbf{Theorem}: If $w, x, y, z$ are non-zero integers where $w$ divides $x$ and $y$ divides $z$, then $wy$ divides $xz$.

For each proof attempt of the theorem, explain where the proof uses invalid reasoning or skips essential steps.
\begin{enumerate}
    \item By assumption, $w$ divides $x$, so $x = kw$ for some integer $k$ and $w \neq 0$. Similarly, $z = ky$ for some integer $k$ and $y \neq 0$. Plug in the expression $kw$ for $x$ and $ky$ for $z$ in the expression $xz$ to get
    \[xz=(kw)(ky)=(k^2)(wy)\]
    Since $k$ is an integer, then $k^2$ is also an integer. Since $w \neq 0$ and $y \neq 0$, then $wy \neq 0$. Since $xz$ equals $wy$ times an integer and $wy \neq 0$, then $wy$ divides $xz$.
    
    \begin{description}
        \item[Answer:].\\
        Here we are saying that $x=kw$ and $z=ky$. However since we have different values of $x,w,z,y$, we cannot use the same value of $k$. We would have to rewrite the definitions to something like $x=kw$ and $z=jy$, where $k \in Z$ and $j \in Z$.
    \end{description}

    \item By assumption, $w$ divides $x$, so $x = kw$ for some integer $k$ and $w \neq 0$. Similarly, $z = jy$ for some integer $j$ and $y \neq 0$. Plug in the expression $kw$ for $x$ and $jy$ for $z$ in the expression $xz$ to get
    \[xz=(kw)(jy)\]
    Since $w \neq 0$ and $y \neq 0$, then $wy \neq 0$. Since $xz$ equals $wy$ times an integer and $wy \neq 0$, then $wy$ divides $xz$.
    
    \begin{description}
        \item[Answer:].\\
        There is a step missing here to show that since $k$ and $j$ are both integers, the product $kj$ will also be an integer. So this means that $(kw)(jy)$ is equal to $(wy)$ times an integer ($kj$).
    \end{description}

    \item By assumption, $w$ divides $x$, so $x = kw$ and $w \neq 0$. Similarly, $z = jy$ and $y \neq 0$. Plug in the expression $kw$ for $x$ and $jy$ for $z$ in the expression $xz$ to get
    \[xz=(kw)(jy)=(kj)(wy)\]
    Since $k$, $j$ are integers, then $kj$ is an integer. Since $w \neq 0$ and $y \neq 0$, then $wy \neq 0$. Since $xz$ equals $wy$ times an integer and $wy \neq 0$, then $wy$ divides $xz$.
    \begin{description}
        \item[Answer:].\\
        Here the definitions $x=kw$ and $y=jy$ are given, but $k$ and $j$ are never defined. Since we are concluding that $wy$ divides $xz$, we need to state that we are assuming $k$ and $j$ are integers.
    \end{description}
\end{enumerate}

\quad

\item Consider the statement: The product of any integer and an even integer is even.
\begin{enumerate}
    \item Define predicates as necessary and write the symbolic form of the statement using quantifiers.
    \begin{description}
        \item[Answer:].\\
        $p(x)$: x is integer\\
        $q(x)$: x is even\\  
        $\forall x \forall y ((p(x) \land q(y)) \rightarrow q(x * y))$
    \end{description}
    \item Prove or disprove the statement. Specify which proof strategy is used.
    \begin{description}
        \item[Answer:].\\
        $x$ is only even if $x=2m$ where $m$ is some integer.\\
        So lets define:\\
        $m$ is some integer and $n$ is some integer\\
        $n=2j$ where $j$ is some integer\\
        $mn=m(2j)=2(mj)$\\
        Since $m$ and $j$ are both integers and result in an integer when multiplied, our final result of $2(mj)$ must be an even integer.
    \end{description}
\end{enumerate}



\quad

\item Consider the statement: If $x$ and $y$ are integers such that $x > 1$, and $x \mid y$, then $x\nmid (y+1)$.
\begin{enumerate}
    \item Write the symbolic form of the statement using quantifiers.
    \begin{description}
        \item[Answer:].\\
        Let $x,y \in Z$\\
        $x > 1 \land x \mid y \rightarrow x \nmid (y+1)$\\
    \end{description}
    \item Prove or disprove the statement. Specify which proof strategy is used.
    \begin{description}
        \item[Answer:].\\
        
    \end{description}
\end{enumerate}

\quad

\item Consider the statement: Every positive integer can be written as the sum of the squares of three integers.
\begin{enumerate}
    \item Write the symbolic form of the statement using quantifiers.
    \begin{description}
        \item[Answer:].\\
        $(\forall n \in \mathbb{Z}^+)(\exists x,y,z \in \mathbb{Z})(n=x^2+y^2+z^2)$
    \end{description}
    \item Prove or disprove the statement. Specify which proof strategy is used.
    \begin{description}
        \item[Answer:].\\
        Proof by counter-example:\\
        Let $n=7$\\
        Then $7=x^2+y^2+z^2$ where $x,y,z$ are all integers.\\
        Now, $x^2 \leq 7, y^2 \leq 7, z^2 \leq 7$\\
        The squares of integers less than or equal to 7 are 0,1,4.\\
        It is not possible to result in 7 as a sum of these three integers.\\
        Therefore, the positive integer 7 cannot be written as a sum of the squares or three integers.
    \end{description}
\end{enumerate}

\quad



\end{enumerate}



\end{document}