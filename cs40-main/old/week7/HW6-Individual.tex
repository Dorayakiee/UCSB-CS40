\documentclass[12pt, oneside]{article}



\documentclass{article}
\usepackage[letterpaper, scale=0.89, centering]{geometry}
\usepackage{amssymb,amsmath,pifont,amsfonts,comment,enumerate}
\usepackage{currfile,xstring,hyperref,tabularx,graphicx,wasysym}
\usepackage[labelformat=empty]{caption}
\usepackage[dvipsnames,table]{xcolor}
\usepackage{multicol,multirow,array,listings,tabularx,lastpage,textcomp,booktabs}



% NOTE: This environment is credit @pnpo (https://tex.stackexchange.com/a/218450)
\lstnewenvironment{algorithm}[1][] %defines the algorithm listing environment
{   
    \lstset{ %this is the stype
        mathescape=true,
        frame=tB,
        numbers=left, 
        numberstyle=\tiny,
        basicstyle=\rmfamily\scriptsize, 
        keywordstyle=\color{black}\bfseries,
        keywords={,procedure, div, for, to, input, output, return, datatype, function, in, if, else, foreach, while, begin, end, }
        numbers=left,
        xleftmargin=.04\textwidth,
        #1
    }
}
{}
\lstnewenvironment{java}[1][]
{   
    \lstset{
        language=java,
        mathescape=true,
        frame=tB,
        numbers=left, 
        numberstyle=\tiny,
        basicstyle=\ttfamily\scriptsize, 
        keywordstyle=\color{black}\bfseries,
        keywords={, int, double, for, return, if, else, while, }
        numbers=left,
        xleftmargin=.04\textwidth,
        #1
    }
}
{}

\newcommand\abs[1]{\lvert~#1~\rvert}
\newcommand{\st}{\mid}

\newcommand{\A}[0]{\texttt{A}}
\newcommand{\C}[0]{\texttt{C}}
\newcommand{\G}[0]{\texttt{G}}
\newcommand{\U}[0]{\texttt{U}}

\newcommand{\cmark}{\ding{51}}
\newcommand{\xmark}{\ding{55}}

\setlength{\parindent}{0em}
\setlength{\parskip}{1em}

\hypersetup{
    colorlinks=true,
    linkcolor=blue,
    filecolor=magenta,      
    urlcolor=cyan,
    pdftitle={Sharelatex Example},
    bookmarks=true,
    pdfpagemode=FullScreen,
}
\usepackage{graphicx} % Required for inserting images

\title{HW1}
\author{Zixuan Chen}
\date{June 2024}

\begin{document}

\maketitle

\section{Introduction}

\end{document}


\usepackage{enumitem}

\title{HW6 Individual}
\author{CS40 Fall'21\\\\\\
Bharat Kathi (5938444)}
\date{Due: Monday, Nov 15, 2021 at 10:00PM on Gradescope}
\begin{document}
\maketitle

{\bf Integrity reminders for individual homeworks}
\begin{itemize}
\item ``Individual homeworks'' must be solely your own work. 
\item You may not collaborate on individual homeworks with anyone or seek help from online tutors or entities outside the class.
\item You may ask questions about the homework in office hours (of the instructor, TAs, and/or tutors) and 
on Piazza.  However, the staff will only answer clarifying questions on these homeworks. You \emph{cannot} use any online resources about the course content other than the text
book and class material from this quarter.
\item Do not share written solutions or partial solutions for homework with other students. Doing so would dilute their learning experience and detract from their success in the class.
\end{itemize}

You will submit this assignment via Gradescope
(\href{https://www.gradescope.com}{https://www.gradescope.com}) in the assignment called ``HW4-Individual''.

\newpage

{\bf Practice three useful proof strategies for comparing two sets:}

\begin{center}
\begin{tabular}{c|c|c}
{\bf  Statement to Prove} & {\bf Logic Form} &{\bf Proof Strategy}\\
\hline 
$A \subseteq B$ & $\forall x  (x \in A  \to x  \in B)$ &For arbitrary $x \in A$, show that $x \in B$ \\
$A \nsubseteq B$ &$\exists x  (x \in A  \land x  \notin B) $ & Find an element $x \in A$, such that $x \notin B$ \\
$A = B$ & $\forall x  ( x\in A \leftrightarrow x \in B)$ & Show that $A \subseteq B$ and $B \subseteq A$ \\
\end{tabular}
\end{center}

\section*{Assigned Questions}
\begin{enumerate}
\item Show that $A \nsubseteq B$, where $A$ and $B$ are defined as follows:
\[
\begin{array}{l}
A = \{3k + 1 \mid k \in \mathbb{N}\}, \\
B = \{4k + 1 \mid k \in \mathbb{N}\}. \\
\end{array}
\]

\begin{description}
    \item[Answer:] .\\
        $k \in \mathbb{N}$\\
        So by substituting in the first 4 values of $k$, we get:\\
        $A = \{ 4,7,10,13,... \}$\\
        $B = \{ 5,9,13,17,... \}$\\
        Since there are some values in $A$ that are not in $B$, we can conclude that $A \nsubseteq B$.
\end{description}

\item Show that $A = B$, where $A$ and $B$ are defined as follows:
\[
\begin{array}{l}
A = \{x\in\mathbb{Z} \mid x \textrm{ is a prime number and } 12 \leq x \leq 18\}, \\
B = \{x\in\mathbb{Z} \mid x = 4k + 1\textrm{ for some } k \in \{3, 4\}\}. \\
\end{array}
\]

\begin{description}
    \item[Answer:] .\\
        The only primes in between 12 and 18 are 13 and 17.\\
        Therefore, $A = \{ 13,17 \}$.\\
        Since $k \in \{3,4\}$, we can simplify $B$ as:\\
        $B = \{4(3)+1,4(4)+1\}$\\
        $B = \{13,17\}$\\
        From this we can see that $A=B$.
\end{description}

\item Let $A = \{6k + 5 \mid k \in \mathbb{N}\}$ and $B = \{3k + 2 \mid k \in \mathbb{N}\}$. Show that $A \neq B$.

\begin{description}
    \item[Answer:] .\\
        $k \in \mathbb{N}$\\
        So by substituting in the first 4 values of $k$, we get:\\
        $A = \{11,17,23,29,...\}$\\
        $B = \{5,8,11,14,...\}$\\
        Since there are some values in $A$ that are not in $B$, we can conclude that $A \neq B$.
\end{description}

\item Let $W = \mathcal{P}(\{1,2,3,4,5\})$. Consider the following statement and attempted proof:
\label{proof_powerset2}

$$\forall A \in W \, \forall B \in W ~\left(~((A \cap B) \subseteq A) \to (A \subseteq B)~\right)$$

\begin{quote}
(1) Towards a universal generalization argument, {\bf choose arbitrary} $A \in W, B \in W$.

(2) We need {\bf to show} $((A \cap B) \subseteq A) \to (A \subseteq B)$.

(3) Towards a proof of the conditional by the contrapositive, {\bf assume} $A \subseteq B$. We need {\bf to show} 
that $(A \cap B) \subseteq A$.

(4) By definition of subset inclusion, this means we need {\bf to show} 
$\forall x ~( x \in A \cap B \to  x \in A  ~)$.


(5) Towards a universal generalization, {\bf choose an arbitrary} $x$; we need {\bf to show} that \\
$x \in A \cap B \to x \in A$.

(6) Towards a direct proof, {\bf assume} $x \in A \cap B$. We need {\bf to show} $x \in A$.

(7) By definition of set intersection, we have that $x \in A \land x \in B$.

(8) By the definition of conjunction, $x \in A$, which is  what we needed to show.

\hfill{$\blacksquare$}

\end{quote}



\begin{enumerate}
    \item Demonstrate that this statement is invalid by providing and justifying a {\bf counterexample} (disproving the statement).
    \begin{description}
        \item[Answer:] .\\
            Let $A = \{2,3\}$\\
            Let $B=\{2\}$\\
            Then, we get that $A \cap B = \{2\}$\\
            Therefore $A \cap B \subseteq A$ but $A \subsetneq B$
    \end{description}
    \item  Explain why the attempted proof is invalid by identifying in which step a definition or proof strategy is used incorrectly, and describing how the definition or proof strategy was misused.
    \begin{description}
        \item[Answer:] The third step is incorrect. The contrapositive of $p \rightarrow q$ should be $\lnot q \rightarrow \lnot p$. The given proof is assuming $q$ in order to prove $p$.
    \end{description}
\end{enumerate}


\item Give a recursive definition for each of the following sets $S$. Each set $S$ will be a subset of the set containing all binary strings. A string $x$ belongs to the recursively defined set $S$ if and only if $x$ has each of the following properties (provide a different definition for each part). Note that in each case, you may provide multiple rules for the recursive step of your definition.


\begin{enumerate}
    \item The set S consists of all strings (including the empty string) that have an even number of $1$'s but may have an even or odd number of zeros.
    \begin{description}
        \item[Answer:] If x belongs to S then x11 or 11x also belongs to S.
    \end{description}
    \item The set $S$ consists of all strings (including the empty string) that have the same number of 0's and 1's.
    \begin{description}
        \item[Answer:] If x belongs to S, then 0S1 or 1S0 also belongs to S.
    \end{description}
\end{enumerate}

\item Apply your recursive definitions for $S$ for each part of the previous question to construct all elements of $S$ with length less than or equal to $4$. Provide your answer in roster notation.

\begin{description}
    \item[Answer:] .\\
    $S = \{0,11,101,011\}$
\end{description}

\end{enumerate}

\end{document}