\documentclass[12pt, oneside]{article}

\usepackage[letterpaper, scale=0.89, centering]{geometry}
\usepackage{amssymb,amsmath,pifont,amsfonts,comment,enumerate}
\usepackage{currfile,xstring,hyperref,tabularx,graphicx,wasysym}
\usepackage[labelformat=empty]{caption}
\usepackage[dvipsnames,table]{xcolor}
\usepackage{multicol,multirow,array,listings,tabularx,lastpage,textcomp,booktabs}



% NOTE: This environment is credit @pnpo (https://tex.stackexchange.com/a/218450)
\lstnewenvironment{algorithm}[1][] %defines the algorithm listing environment
{   
    \lstset{ %this is the stype
        mathescape=true,
        frame=tB,
        numbers=left, 
        numberstyle=\tiny,
        basicstyle=\rmfamily\scriptsize, 
        keywordstyle=\color{black}\bfseries,
        keywords={,procedure, div, for, to, input, output, return, datatype, function, in, if, else, foreach, while, begin, end, }
        numbers=left,
        xleftmargin=.04\textwidth,
        #1
    }
}
{}
\lstnewenvironment{java}[1][]
{   
    \lstset{
        language=java,
        mathescape=true,
        frame=tB,
        numbers=left, 
        numberstyle=\tiny,
        basicstyle=\ttfamily\scriptsize, 
        keywordstyle=\color{black}\bfseries,
        keywords={, int, double, for, return, if, else, while, }
        numbers=left,
        xleftmargin=.04\textwidth,
        #1
    }
}
{}

\newcommand\abs[1]{\lvert~#1~\rvert}
\newcommand{\st}{\mid}

\newcommand{\A}[0]{\texttt{A}}
\newcommand{\C}[0]{\texttt{C}}
\newcommand{\G}[0]{\texttt{G}}
\newcommand{\U}[0]{\texttt{U}}

\newcommand{\cmark}{\ding{51}}
\newcommand{\xmark}{\ding{55}}

\setlength{\parindent}{0em}
\setlength{\parskip}{1em}

\hypersetup{
    colorlinks=true,
    linkcolor=blue,
    filecolor=magenta,      
    urlcolor=cyan,
    pdftitle={Sharelatex Example},
    bookmarks=true,
    pdfpagemode=FullScreen,
}
\usepackage{enumitem}

\title{HW8 Collaborative}
\author{CS40 Fall'21\\\\\\
Benjamin Cruttenden (4672440)\\\
Bharat Kathi (5938444)\\\
Sean Oh (4824231)\\\
Marco Wong (4589198)}
\date{Due: Monday, Nov 29, 2021 at 10:00PM on Gradescope}
\begin{document}
\maketitle

{\bf In this assignment,}

You will have more practice with induction and other proof strategies.


{\bf For all HW assignments:}

Please see the instructions and policies for assignments on the class website and on the writeup for HW1.  In particular, these policies address
\begin{multicols}{2}
\begin{itemize}
\item Collaboration policy
\item Where to get help
\item Typing your solutions
\item Expectations for full credit
\end{itemize}
\end{multicols}


You will submit this assignment via Gradescope
(\href{https://www.gradescope.com}{https://www.gradescope.com}) in the assignment called ``HW8''.


In your proofs and disproofs of statements below, justify each  step
by reference to the proof strategies
we have discussed so far, and/or to relevant definitions and calculations. We include only induction-related strategies here; you can and should refer to past material to identify others.

{\bf Proof by Mathematical Induction}: To prove a universal quantification over the set of  all integers greater than  or  equal to some base integer $b$:

\begin{tabularx}{\textwidth}{l X}
    Basis Step: & Show the statement holds for $b$. \\
    Recursive Step: & Consider an arbitrary integer $n$ greater than or  equal to  $b$, assume
    (as the {\bf induction hypothesis})  that the property holds  for $n$, and use  this and
    other facts to  prove that  the property holds for $n+1$.
\end{tabularx}

{\bf Proof by Strong Induction} To prove that a universal quantification over the set of all integers greater than or equal to some  base integer $b$ holds,  pick a  fixed nonnegative integer  $j$ and then: \hfill 

\begin{tabularx}{\textwidth}{l X}
    Basis Step: & Show the statement holds for $b$, $b+1$, \ldots, $b+j$. \\
    Recursive Step: & Consider an arbitrary integer $n$ greater than or  equal to  $b+j$, assume
    (as the {\bf strong  induction hypothesis})  that the property holds  for {\bf each of} $b$, $b+1$, \ldots, $n$, 	
    and use  this and
    other facts to  prove that  the property holds for $n+1$.
\end{tabularx}
\newpage


{\bf Assigned questions}

\begin{enumerate}

\item Assume $L$ is the set of linked lists defined recursively in HW7 and $toNum$ is a function from $L \to \mathbb{N}$ defined recursively as follows (also covered in Week 9 Monday's lecture):

\[
\begin{array}{ll}
\textrm{Basis Step: } &   toNum([])  = 0\\

\textrm{Recursive Step: } & \textrm{If } n \in \mathbb{N} \textrm{ and } l \in L \textrm{, then } toNum((n, l))  = 2^n \cdot 3^{toNum(l)}\\
\end{array}
\]


Prove that $toNum$ is one-to-one.

\begin{description}
    \item[Answer:] .\\
    $\forall l1 \in L \forall l2 \in  L (tonum(l1) = tonum(l2) \rightarrow (l1 = l2))$\\
    Direct proof:\\
    Assume tonum(l1) = tonum(l2)\\
    Then: $2^n*3^{tonum(l1')} = 2^n*3^{tonum(l2')}$ // Divide both sides by $2^n$\\
    $3^{tonum(l1')} = 3^{tonum(l2')}$\\\\
    $N1 = n2$\\
    $tonum(l1') = tonum(l2')$\\\\
    Induction: on l1 choose arbitrary l2\\
    Base case: l1 = []\\
    WTS: tonum([]) = tonum(l2) $\rightarrow$ [] = l2\\
    Direct proof:\\
    Assume tonum([]) = tonum(l2)\\
    0 = tonum(l2) by basis step\\
    L2 = [] by basis step\\\\
    Inductive Case: l1 = (n1,l1’). Choose arbitrary (n2,l2)\\
    Assume as inductive hypothesis: $tonum(l1') = tonum(l2) \rightarrow (l1' = l2)$\\
    WTS: $tonum((n1,l1')) = tonum(n2,l2) -> ((n1,l1') = (n2,l2))$\\
    Direct proof:\\
    Assume tonum((n1,l1’)) = tonum(n2,l2)\\
    $2^{n1}*3^{tonum(l1')} = 2^{n2}*3^{tonum(l2)}$  //  by recursive step\\
    $2^{n1-n2}*3^{tonum(l1')-tonum(l2)} = 1$\\
    n1=n2\\
    tonum(l1’) = tonum(l2)\\
    $\Rightarrow$ (l1’ = l2)  //  by inductive hypothesis\\
    $\Rightarrow $((n1,l1) = (n2,l2))  // by adding n1 and n2 to each side since n1 = n2
\end{description}

\item Define $P(n)$ to be the assertion that:
\[ \sum_{j=1}^{n} j^2 = n(n+1)(2n+1)/6 \]
Answer the questions that follow:
\begin{enumerate}
    \item Verify that $P(3)$ is true, and then express $P(k)$ and $P(k + 1)$
    
    \begin{description}
        \item[Answer:] .\\
        Verify P(3) $LHS = \sum_{j=1}^{3} j^2 = 3^2 + 2^2 + 1^2 = 9 + 4 + 1 = 14$\\
        $RHS = 3(3 + 1)(2(3) + 1)/6 = 3*4*7/6 = 14$\\
        LHS = RHS verifying P(3)\\
        $P(k): \sum_{j=1}^{k} k^2 = k(k + 1)(2(k) + 1)/6$\\
        $P(k): \sum_{j=1}^{k+1} (k+1)^2 = (k+1)(k + 1 + 1)(2(k+ 1) + 1)/6$ \\
    \end{description}

    \item What is the basis step for an inductive proof of $\forall n \in \mathbb{Z^+}(P(n))$
    
    \begin{description}
        \item[Answer:] .\\
        Basis Step, n = 1 so P(1) is the assertion that\\
        $P(k): \sum_{j=1}^{1} j^2 = 1(1 + 1)(2(1) + 1)/6  $\\
        Evaluating: $LHS = 1^2 = 1$\\
        RHS = 1(2)(3)/6 = 6/6 = 1\\
        LHS = RHS so n = 1 is true
    \end{description}
    
    \item What would be the inductive hypothesis? What must be proven in the inductive step?
    
    \begin{description}
        \item[Answer:] .\\
        Inductive hypothesis would be that the assertion is true for n = k\\
        So IH is P(k): $\sum_{j=1}^{k+1} k^2 = (k+1)(k + 1 + 1)(2(k+ 1) + 1)/6 $\\
        And what must be prove is n = k + 1\\
        So WTS is P(k+1): $\sum_{j=1}^{k+1} (k+1)^2 = (k+1)(k + 1 + 1)(2(k+ 1) + 1)/6 $\\
        $= k^3/3 + 3k^2/2 + 13k/6 + 1$
    \end{description}

    \item Write the complete inductive proof for the provided assertion by combining all your answers from the previous parts
\end{enumerate}

\begin{description}
    \item[Answer:] .\\
    So by inductive proof over n\\
    Basis Step, n = 1 so P(1) is the assertion that\\
    $\sum_{j=1}^{1} j^2 = 1(1 + 1)(2(1) + 1)/6  $\\
    Evaluating: LHS = $1^2$ = 1\\
    RHS = 1(2)(3)/6 = 6/6 = 1\\
    LHS = RHS so n = 1 is true\\\\
    Inductive step: Now with that being true we form the IH of n = k being true,\\
    So IH is P(k): $\sum_{j=1}^{k} (k)^2 = k(k + 1)(2(k) + 1)/6  = k^3/3 + k^2/2 + k/6$\\
    We WTS that its true for n = k + 1 which is\\
    P(k+1): $\sum_{j=1}^{k+1} (k+1)^2 = (k+1)(k + 1 + 1)(2(k+ 1) + 1)/6$\\
    Evaluating LHS = $\sum_{j=1}^{k+1} (k+1)^2$\\
    $= (k+1)^2 + \sum_{j=1}^{k} k^2$\\
    $= k^2 + 2k + 1 + \sum_{j=1}^{k} k^2$\\
    RHS = (k+1)(k + 1 + 1)(2(k+ 1) + 1)/6\\
    $= k^3/3 + 3k^2/2 + 13k/6 + 1$\\
    $= k^3/3 + k^2/2 + k^2 + k/6 + 2k + 1$\\
    $= k^2 + 2k + 1 + \sum_{j=1}^{k} k^2$\\
    LHS = RHS so proving it
\end{description}

\item ({\it Graded for correctness in evaluating statement 
and for fair effort completeness in the justification}) Consider the functions $f_a: \mathbb{N} \to \mathbb{N}$ and $f_b: \mathbb{N} \to \mathbb{N}$ 
defined recursively by 
\[
f_a(0) = 0 \qquad\text{and for each $n \in \mathbb{N}$,} \qquad f_a(n+1) = f_a(n) + 2n + 1
\]
\[
f_b(0) = 0 \qquad\text{and for each $n \in \mathbb{N}$,} \qquad f_b(n+1) = 2f_b(n)
\]
Which of these two functions (if any) equals $2^n$ and which of these functions (if any) equals $n^2$? Use induction to prove the equality or use counterexamples to disprove it.

\item ({\it Graded for correctness}) Prove the following statement:
$$\exists n_0 \in \mathbb{N} \, \forall n \in \mathbb{Z}^{\geq n_0} \, (n^2 < 2^n)$$
In your proof, you may use the following lemma:
$$\exists n_0 \in \mathbb{N} \, \forall n \in \mathbb{Z}^{\geq n_0} \, (1+2n~ < n^2)$$

\newpage
\rule{0.5\textwidth}{.4pt}

{\it Proof of lemma. This proof can also be used as reference for a possible approach for the statement you are trying to prove:} 

To prove the existential claim, consider the witness $n_0 = 3$. We will prove that
$$ \forall n \in \mathbb{Z}^{\geq 3} \, (1+2n~ < n^2)$$
using mathematical induction.

{\bf Basis step} For the basis step, we need to show that $1+2 \cdot 3 ~<~3^2$. Evaluating: $1+2 \cdot 3 = 1 + 6 =7$
and $3^2 = 9$. Since $7< 9$, the basis step is complete.

{\bf Recursive step} Consider arbitrary integer $n$ that is greater than or equal to $3$. Assume, as the induction hypothesis, 
that $1 + 2n ~<~ n^2$. We need to show that $1+2(n+1)~ < (n+1)^2$. Calculating: 
\begin{alignat*}{2}
(n+1)^2 &= (n+1) (n+1) = n^2 + 2n + 1 && \\
&> (1 + 2n) + 2n +1 &&\text{by the induction hypothesis} \\
&> 2n + 2n + 1 && \text{since $1 > 0$} \\
&> 2n + 2 \cdot 1 + 1 &&\text{since $n > 1$ by assumption that $n \geq 3$}\\
&= 2(n+1) +1 = 1 + 2(n+1) &&\text{as required to complete the recursive step.}
\end{alignat*}

Thus, the universal quantification was proved using mathematical induction and so the witness $n_0 = 3$
proves the existential. \hfill{$\blacksquare$}

\rule{0.5\textwidth}{.4pt}

\item ({\it Graded for fair effort completeness}) 
Can the statement you proved above be used to prove or disprove the following statement? Why or why not?
$$\exists n_0 \in \mathbb{N} \, \forall n \in \mathbb{Z}^{\geq n_0} \, (2^n < n^2)$$

\item Prove the following upper bound for the given recurrence relation using strong induction.

Define the sequence $\{a_n\}$ as follows:\\

    $a_1 = a_2 = a_3 = 1$\\
    $a_n = a_{n-1} + a_{n-2} + a_{n-3}$, for $n \geq 4$

Prove that $$\forall n \in \mathbb{Z}^{\geq 1} \, (~a_n \leq 2^n~)$$

\newpage
\item Use induction to prove that the following algorithm is correct. Binary strings are the set of all strings of length 0 or more made up of characters from the set $\{0, 1\}$.
% xiaoyong, please replace the pseudocode below with solution given in zybook exercise 8.4.8(a)


\begin{algorithm}[caption={Recursively computing the set of all binary strings of a fixed length $n$}]
procedure $\textit{StringSet}$($n$:a non-negative integer)

$S$ := $\emptyset$
if $n$ = $0$
    Add $\lambda$ to $S$
    return $(S)$ 
end if
$T$ := $StringSet(n-1)$
for every $x\in T$
    $y$ := $0x$
    Add $y$ to $S$
    $y$ := $1x$
    Add $y$ to $S$
end for
return $(S)$ {output is $S$}

    
\end{algorithm}



\item Let $W = \mathcal{P}(\{1,2,3,4,5\})$. 
\label{proof_powerset}

\rule{0.5\textwidth}{.4pt}

{\it Sample response that can be used as reference for the detail expected 
in your answer for parts (a) and (b) below:} 

To give an example element in the set 
$\{ X \in W ~|~ 1 \in X \} \cap \{ X \in W ~|~  2 \in X \}$,
consider $\{ 1,2\}$. To prove that this is in the set, by definition of intersection, we need to show
that $\{1,2\} \in \{ X \in W ~|~ 1 \in X \}$ and that $\{1,2\} \in \{ X \in W ~|~ 2 \in X \}$.
\begin{itemize}
\item By set builder notation, elements in $\{ X \in W ~|~ 1 \in X \}$ have to be elements of $W$ which have $1$ as an element. By definition of power set, elements of $W$ are subsets of $\{1,2,3,4,5\}$. Since
each element in $\{1,2\}$ is an element of $\{1,2,3,4,5\}$, $\{1,2\}$ is a subset of $\{1,2,3,4,5\}$ 
and hence is an element of $W$. Also, by roster method, $1 \in \{1,2\}$. Thus, $\{1,2\}$ satisfies the 
conditions for membership in $\{ X \in W ~|~ 1 \in X \}$.
\item Similarly, by set builder notation, elements in $\{ X \in W ~|~ 2 \in X \}$ have to be elements of $W$ 
which have $2$ as an element. 
By definition of power set, elements of $W$ are subsets of $\{1,2,3,4,5\}$. Since
each element in $\{1,2\}$ is an element of $\{1,2,3,4,5\}$, $\{1,2\}$ is a subset of $\{1,2,3,4,5\}$ 
and hence is an element of $W$. Also, by roster method, $2 \in \{1,2\}$. Thus, $\{1,2\}$ satisfies the 
conditions for membership in $\{ X \in W ~|~ 2 \in X \}$.
\end{itemize}

\rule{0.5\textwidth}{.4pt}


\begin{enumerate}
\item Give two example elements in 
\[
W \times W.
\]
Justify your examples by explanations that include references to the relevant definitions.

\item Give one example element in 
\[
\mathcal{P}(W)
\]
that is {\bf not} equal to $\emptyset$ or to $W$. Justify your example by an explanation that includes references to the relevant definitions.
\end{enumerate}

\item We define the following function: $$f: \{0,1\}^4 \rightarrow \{0,1\}^3,$$ where the output of $f$ is obtained by taking the input string and dropping the first bit. For example $f(1011) = 011$. Indicate whether the $f$ is onto, one-to-one, neither or both. If the function is not onto or not one-to-one, give an example showing why.

\end{enumerate}


\section*{Attributions}

Thanks to \href{http://cseweb.ucsd.edu/~minnes/}{Mia Minnes} and \href{https://jpolitz.github.io/}{Joe Politz} for the original version of the instructions and select questions. All materials created by them is licensed under a \href{http://creativecommons.org/licenses/by-nc/4.0/}{Creative Commons Attribution-Non Commercial 4.0} International License.

\end{document}