\documentclass[12pt, oneside]{article}

\usepackage[letterpaper, scale=0.89, centering]{geometry}
\usepackage{amssymb,amsmath,pifont,amsfonts,comment,enumerate}
\usepackage{currfile,xstring,hyperref,tabularx,graphicx,wasysym}
\usepackage[labelformat=empty]{caption}
\usepackage[dvipsnames,table]{xcolor}
\usepackage{multicol,multirow,array,listings,tabularx,lastpage,textcomp,booktabs}



% NOTE: This environment is credit @pnpo (https://tex.stackexchange.com/a/218450)
\lstnewenvironment{algorithm}[1][] %defines the algorithm listing environment
{   
    \lstset{ %this is the stype
        mathescape=true,
        frame=tB,
        numbers=left, 
        numberstyle=\tiny,
        basicstyle=\rmfamily\scriptsize, 
        keywordstyle=\color{black}\bfseries,
        keywords={,procedure, div, for, to, input, output, return, datatype, function, in, if, else, foreach, while, begin, end, }
        numbers=left,
        xleftmargin=.04\textwidth,
        #1
    }
}
{}
\lstnewenvironment{java}[1][]
{   
    \lstset{
        language=java,
        mathescape=true,
        frame=tB,
        numbers=left, 
        numberstyle=\tiny,
        basicstyle=\ttfamily\scriptsize, 
        keywordstyle=\color{black}\bfseries,
        keywords={, int, double, for, return, if, else, while, }
        numbers=left,
        xleftmargin=.04\textwidth,
        #1
    }
}
{}

\newcommand\abs[1]{\lvert~#1~\rvert}
\newcommand{\st}{\mid}

\newcommand{\A}[0]{\texttt{A}}
\newcommand{\C}[0]{\texttt{C}}
\newcommand{\G}[0]{\texttt{G}}
\newcommand{\U}[0]{\texttt{U}}

\newcommand{\cmark}{\ding{51}}
\newcommand{\xmark}{\ding{55}}

\setlength{\parindent}{0em}
\setlength{\parskip}{1em}

\hypersetup{
    colorlinks=true,
    linkcolor=blue,
    filecolor=magenta,      
    urlcolor=cyan,
    pdftitle={Sharelatex Example},
    bookmarks=true,
    pdfpagemode=FullScreen,
}
\usepackage{enumitem}

\title{HW3 Individual}
\author{CS40 Fall'21\\\\\\
Bharat Kathi (5938444)}
\date{Due: Monday, Oct 25, 2021 at 10:00PM on Gradescope}

\begin{document}
\maketitle

{\bf Integrity reminders for individual homeworks}
\begin{itemize}
\item ``Individual homeworks'' must be solely your own work. 
\item You may not collaborate on individual homeworks with anyone or seek help from online tutors or entities outside the class.
\item You may ask questions about the homework in office hours (of the instructor, TAs, and/or tutors) and 
on Piazza.  However, the staff will only answer clarifying questions on these homeworks. You \emph{cannot} use any online resources about the course content other than the text
book and class material from this quarter.
\item Do not share written solutions or partial solutions for homework with other students. Doing so would dilute their learning experience and detract from their success in the class.
\end{itemize}

You will submit this assignment via Gradescope
(\href{https://www.gradescope.com}{https://www.gradescope.com}) in the assignment called ``HW3-Individual''.

\subsection*{Summary of Proof Strategies (so far)}
In your proofs and disproofs of statements below, justify each  step
by reference to  a component of the  following proof  strategies
we  have discussed so far, and/or to relevant definitions and calculations.
\begin{itemize}
    \item A counterexample can be used to prove that  $\forall x P(x)$ is {\bf false}.
    \item  A witness can be used  to  prove that  $\exists x P(x)$ is {\bf true}.
    \item {\bf Proof of universal by exhaustion}: To prove that $\forall x \, P(x)$
is true when $P$ has a finite domain, evaluate the predicate at {\bf each} domain element to confirm that it is always T.
    \item  {\bf Proof by universal generalization}: To prove that $\forall x \, P(x)$
is true, we can take an arbitrary element $e$ from the domain and show that $P(e)$ is true, without making any assumptions about $e$ other than that it comes from the domain.
    \item To  prove  that $\exists x P(x)$ is {\bf false}, write the universal statement that is logically equivalent to its negation and then prove it true using universal generalization.
   
    \item {\bf Proof of Conditional by Direct Proof}: To prove that the implication $p \to q$ is true, we can assume $p$ is true and use that assumption to show $q$ is true.

\end{itemize}

\newpage
\section*{Assigned Questions}

\begin{enumerate}

\item ({\it Graded for correctness}) Is the following argument valid or invalid? Prove your answer by replacing each proposition with a variable to obtain the form of the argument. Then prove that the form is valid or invalid using truth tables. 

I will buy a new car and a new house only if I get a job.\\
I am not going to get a job.\\
I will buy a new house.\\
Therefore, I will not buy a new car.

\begin{description}
    \item[Answer:] .\\
    p: I will buy a new car.\\
    q: I will buy a new house\\
    r: I will get a job.\\\\

    $r \rightarrow (p \land q)$ (premise 1)\\
    $\lnot r$ (premise 2)\\
    $q$ (premise 3)\\
    $\therefore \lnot p$ (conclusion)\\\\
    \begin{displaymath}
        \begin{array}{|c c c|c c c|c|}
        p & q & r & r \rightarrow (p \land q) & \lnot r & q & \lnot p\\
        \hline
        T & T & T & T & F & T & F\\
        T & T & F & T & T & T & F\\
        T & F & T & F & F & F & F\\
        T & F & F & T & T & F & F\\
        F & T & T & F & F & T & T\\
        F & T & F & T & T & T & T\\
        F & F & T & F & F & F & T\\
        F & F & F & T & F & F & T\\
        \end{array}
    \end{displaymath}\\
    From this truth table, we can conclude that the given argument is valid since that when all the premises are true, the conclusion is also true (row 6).
\end{description}


\item({\it Graded for correctness}) Imagine that the friend from last homework comes back to assert that:

The statement

\[
\forall x (~P(x) \lor  Q(x)~) 
\]

implies the statement

\[
(\forall x P(x)) \lor (\forall x Q(x)) 
\]

This time, instead of simply making an assertion they provide the following mathematical argument:
\begin{quote}

(1) $\forall x(P(x) \lor Q(x))$ \hfill{Premise}

(2) $P(c) \lor Q(c)$\text{ for some $c$ in the domain}
\hfill{Existential instantiation from (1)}

(3) $P(c)$ \hfill{Simplification from (2)}

(4) $\forall x P(x)$ \hfill{Universal generalization from (3)}

(5) $Q(c)$ \hfill{Simplification from (2)}

(6) $\forall xQ(x)$
\hfill{Universal generalization from (5)}

(7) $(\forall xP(x)) \lor (\forall xQ(x))$
\hfill{Addition from (4) and (6)}

\hfill{$\blacksquare$}

\end{quote}

\begin{enumerate}

\item ({\it Graded for correctness\footnote{This means your solution will be
evaluated not only on the correctness of your answers, but on your ability to 
present your ideas clearly and logically. You should explain how you arrived at your conclusions, using 
mathematically sound reasoning. Whether you use formal proof techniques or write a more informal argument for why 
something is true, your answers should always be well-supported. Your goal should be to convince the reader that 
your results and methods are sound.}}) Prove to your friend that they made a mistake by identifying \textbf{all of the error(s)} in their argument and the specific steps where they made each error. Explain the errors clearly enough that a student in CS 40 who may be 
struggling with the material can still follow along with your reasoning.
\begin{description}
    \item[Answer:] .\\
    From the given statement $\forall x (P(x) \lor Q(x))$, we can deduce that $P \land Q \implies P$ or $P \land Q \implies Q$.\\
    However, we cannot deduce $P \lor Q \implies P$ or $P \lor Q \implies Q$ from that statement.\\
    In step (3), the incorrect deduction is made that $P \lor Q \implies P$.\\
    In step (5), the incorrect deduction is made that $P \lor Q \implies Q$.\\

\end{description}
\item ({\it Graded for fair effort completeness}\footnote{This means you will get full credit so long as your submission 
demonstrates honest effort to answer the question. You will not be penalized for incorrect answers.} ) Show that for any choice of $P(x)$ and $Q(x)$, the hypothesis $\exists x (~P(x) \lor  Q(x)~)$ implies the conclusion $(\exists x P(x)) \lor (\exists x Q(x))$
\begin{description}
    \item[Answer:] .\\
    Hypothesis: $\exists x (P(x) \lor Q(x))$\\
    Conclusion: $(\exists x P(x)) \lor (\exists x Q(x))$\\
    Negation of Hypothesis: $\lnot \exists x (P(x) \lor Q(x)) \equiv \forall x \lnot P(x) \land \forall x \lnot Q(x)$\\\\
    (1) $\forall x \lnot P(x) \land \forall x \lnot Q(x)$\\
    (2) $\forall x \lnot P(x)$ - simplification from (1)\\
    (3) $\lnot P(a)$ - Existential Instantiation\\
    (4) $\forall x \lnot Q(x)$ - simplification from (1)\\
    (5) $\lnot Q(a)$ - Existential Instantiation\\
    (6) $\forall x (P(x) \lor Q(x))$ - hypothesis\\
    (7) $P(a) \lor Q(a)$ - Existential Instantiation\\

\end{description}

\end{enumerate}

\item({\it Graded for correctness}) Justify the rule of universal transitivity, which states that if $\forall x(P(x) \to Q(x))$ and $\forall x(Q(x) \to R(x))$ are true, then $\forall x(P(x)\to R(x))$ is true, where the domains of all quantifiers are the same.
\begin{description}
    \item[Answer:] .\\
    Assume $\forall x (P(x) \rightarrow Q(x))$ is true.\\
    This means that for all x, if P(x) is true then Q(x) is also true.\\
    Therefore, if we have some value a, $P(a) \rightarrow Q(a)$ is true.\\\\
    Similary, we assume $\forall x (Q(x) \rightarrow R(x))$ is true.\\
    This means that for all x, if Q(x) is true then R(x) is also true.\\
    Therefore, if we have some value a, $Q(a) \rightarrow R(a)$ is true.\\\\
    Using Hypothetical Syllogism, we can infer that since $P(a) \rightarrow Q(a)$ and $Q(a) \rightarrow R(a)$, then $P(a) \rightarrow R(a)$.\\
    Lastly, we can generalize this into $\forall x (P(x) \rightarrow R(x))$
\end{description}

\item ({\it Graded for correctness}) Assuming that the domains of all quantifiers are the same, use rules of inference to show that if $
\forall x(P(x)\lor Q(x))  \land \forall x((\neg P(x)\land Q(x)) \to R(x))
$ then, $\forall x(\neg R(x) \to P(x))$ \\

\begin{description}
    \item[Answer:] .\\
    (1) $\forall (P(x) \lor Q(x))$ - premise\\
    (2) $\forall x ((\lnot P(x) \land Q(x)) \rightarrow R(x))$ - premise\\
    (3) $(\lnot P(a) \land Q(a)) \rightarrow R(a)$ - Universal instantiation\\
    (4) $\lnot R(a)$ - Premise for direct proof\\
    (5) $\lnot (\lnot P(a) \land Q(a))$ - modus tollens\\
    (6) $P(a) \lor \lnot Q(a)$ - DeMorgan's Law, universal instantiation\\
    (7) $P(a) \lor P(a)$\\
    (8) $P(a)$\\
    (9) $\lnot R(a) \rightarrow P(a)$\\
    (10) $\forall x (\lnot R(x) \rightarrow P(x))$
\end{description}







\end{enumerate}





\end{document}