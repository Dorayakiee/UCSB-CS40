\documentclass[12pt, oneside]{article}



\documentclass{article}
\usepackage[letterpaper, scale=0.89, centering]{geometry}
\usepackage{amssymb,amsmath,pifont,amsfonts,comment,enumerate}
\usepackage{currfile,xstring,hyperref,tabularx,graphicx,wasysym}
\usepackage[labelformat=empty]{caption}
\usepackage[dvipsnames,table]{xcolor}
\usepackage{multicol,multirow,array,listings,tabularx,lastpage,textcomp,booktabs}



% NOTE: This environment is credit @pnpo (https://tex.stackexchange.com/a/218450)
\lstnewenvironment{algorithm}[1][] %defines the algorithm listing environment
{   
    \lstset{ %this is the stype
        mathescape=true,
        frame=tB,
        numbers=left, 
        numberstyle=\tiny,
        basicstyle=\rmfamily\scriptsize, 
        keywordstyle=\color{black}\bfseries,
        keywords={,procedure, div, for, to, input, output, return, datatype, function, in, if, else, foreach, while, begin, end, }
        numbers=left,
        xleftmargin=.04\textwidth,
        #1
    }
}
{}
\lstnewenvironment{java}[1][]
{   
    \lstset{
        language=java,
        mathescape=true,
        frame=tB,
        numbers=left, 
        numberstyle=\tiny,
        basicstyle=\ttfamily\scriptsize, 
        keywordstyle=\color{black}\bfseries,
        keywords={, int, double, for, return, if, else, while, }
        numbers=left,
        xleftmargin=.04\textwidth,
        #1
    }
}
{}

\newcommand\abs[1]{\lvert~#1~\rvert}
\newcommand{\st}{\mid}

\newcommand{\A}[0]{\texttt{A}}
\newcommand{\C}[0]{\texttt{C}}
\newcommand{\G}[0]{\texttt{G}}
\newcommand{\U}[0]{\texttt{U}}

\newcommand{\cmark}{\ding{51}}
\newcommand{\xmark}{\ding{55}}

\setlength{\parindent}{0em}
\setlength{\parskip}{1em}

\hypersetup{
    colorlinks=true,
    linkcolor=blue,
    filecolor=magenta,      
    urlcolor=cyan,
    pdftitle={Sharelatex Example},
    bookmarks=true,
    pdfpagemode=FullScreen,
}
\usepackage{graphicx} % Required for inserting images

\title{HW1}
\author{Zixuan Chen}
\date{June 2024}

\begin{document}

\maketitle

\section{Introduction}

\end{document}


\usepackage{enumitem}

\title{HW3 Collaborative}
\author{CS40 Fall'21\\\\\\
Benjamin Cruttenden (4672440)\\\
Bharat Kathi (5938444)\\\
Sean Oh (4824231)\\\
Marco Wong (4589198)}
\date{Due: Thursday, Oct 21, 2021 at 10:00PM on Gradescope}
\begin{document}
\maketitle

{\bf For all Collaborative HW assignments:}

Collaborative homeworks may be done individually or in groups of up to 4 students. You may switch HW partners for different HW assignments. The lowest HW score will not be included in your overall HW average. Please ensure your name(s) and PID(s) are clearly visible on the first page of your homework submission.

All submitted homework for this class must be typed. Diagrams may be hand-drawn and scanned and included in the typed document. You can use a word processing editor if you like (Microsoft Word, Open Office, Notepad, Vim, Google Docs, etc.) but you might find it useful to take this opportunity to learn LaTeX. LaTeX is a markup language used widely in computer science and mathematics. The homework assignments are typed using LaTeX and you can use the source files as templates for typesetting your solutions.\footnote{To use this template, you will need to copy both the source file (extension \texttt{.tex})  you'll be editing
and the file containing all the ``shortcut" commands we've defined for this class \href{https://drive.google.com/file/d/1FmQvgByKnNjTpIkAUw31TGWYrQZM-HK0/view?usp=sharing})}


{\bf Integrity reminders}
\begin{itemize}
\item Problems should be solved together, not divided up between the partners. The homework is
designed to give you practice with the main concepts and techniques of the course, while getting to know and learn from your classmates.
\item You may not collaborate on homework with anyone other than your group members.
You may ask questions about the homework in office hours (of the instructor, TAs, and/or tutors) and 
on Piazza.  You \emph{cannot} use any online resources about the course content other than the text
book and class material from this quarter -- this is primarily to ensure that we all use consistent notation and
definitions we will use this quarter.
\item Do not share written solutions or partial solutions for homework with other students in the class who are not in your group. Doing so would dilute their learning experience and detract from their success in the class.
\end{itemize}


You will submit this assignment via Gradescope
(\href{https://www.gradescope.com}{https://www.gradescope.com}) in the assignment called ``HW3-Collaborative''.

\subsection*{Summary of Proof Strategies (so far)}
In your proofs and disproofs of statements below, justify each  step
by reference to  a component of the  following proof  strategies
we  have discussed so far, and/or to relevant definitions and calculations.
\begin{itemize}
    \item A counterexample can be used to prove that  $\forall x P(x)$ is {\bf false}.
    \item  A witness can be used  to  prove that  $\exists x P(x)$ is {\bf true}.
    \item {\bf Proof of universal by exhaustion}: To prove that $\forall x \, P(x)$
is true when $P$ has a finite domain, evaluate the predicate at {\bf each} domain element to confirm that it is always T.
    \item  {\bf Proof by universal generalization}: To prove that $\forall x \, P(x)$
is true, we can take an arbitrary element $e$ from the domain and show that $P(e)$ is true, without making any assumptions about $e$ other than that it comes from the domain.
    \item To  prove  that $\exists x P(x)$ is {\bf false}, write the universal statement that is logically equivalent to its negation and then prove it true using universal generalization.
   
    \item {\bf Proof of Conditional by Direct Proof}: To prove that the implication $p \to q$ is true, we can assume $p$ is true and use that assumption to show $q$ is true.

   
\end{itemize}

\newpage
\section*{Assigned Questions}

\begin{enumerate}

\item What rule of inference is used in each of these arguments?
\begin{enumerate}
    \item Kangaroos live in Australia and are marsupials. Therefore, kangaroos are marsupials.
    \begin{description}
        \item[Answer:] Simplification
    \end{description}
    \item It is either hotter than 100 degrees today or the pollution is dangerous. It is less than 100 degrees outside today. Therefore, the pollution is dangerous.
    \begin{description}
        \item[Answer:] Disjunctive Syllogism
    \end{description}
\end{enumerate}
\item What rule of inference is used in each of these arguments?
\begin{enumerate}
    \item Linda is an excellent swimmer. If Linda is an excellent swimmer, then she can work as a lifeguard. Therefore, Linda can work as a lifeguard.
    \begin{description}
        \item[Answer:] Modus Ponens
    \end{description}
    \item If I work all night on this homework, then I can answer all the exercises. If I answer all the exercises, I will understand the material. Therefore, if I work all night on this homework, then I will understand the material.
    \begin{description}
        \item[Answer:] Hypothetical Syllogism
    \end{description}
\end{enumerate}

\item For each of these arguments determine whether the argument is correct or incorrect and explain why. If the argument is correct, state which rule of inference is used, and if it is incorrect, give a reasonable explanation for why it is incorrect.

\begin{enumerate}
    \item Everyone enrolled in the university has lived in a dormitory. Mia has never lived in a dormitory. Therefore, Mia is not enrolled in the university.
    \begin{description}
        \item[Answer:] .\\
        Q(x): enrolled in the university\\
        P(x): lived in dorms\\
        $\forall x (Q(x) \rightarrow P(x))$\\
        $\lnot P(Mia)$\\
        $= \lnot Q(Mia)$\\
        True by universal instantiation and modus tollens
    \end{description}
    \item A convertible car is fun to drive. Isaac’s car is not a convertible. Therefore, Isaac’s car is not fun to drive.
    \begin{description}
        \item[Answer:] .\\
        P(x): is convertible\\
        Q(x): is fun to drive\\
        $\exists x(P(x) \rightarrow Q(x))$\\
        $\lnot P(Isaac)$\\
        $= \lnot Q(Isaac)$\\
        This one is False because If Negative P(Isaac) is True then P(x) has to be false.
        If P(x) is False then Q(x) can be both T or F. If the hypothesis is false then The conclusion doesn’t matter, so it can not be conclusively False.
    \end{description}
    \item Quincy likes all action movies. Quincy likes the movie {\it Eight Men Out}. Therefore, Eight Men Out is an action movie.
    \begin{description}
        \item[Answer:] .\\
        Q(x): x is an action movie\\
        P(x): Quincy liked x movie\\
        $\forall x (P(x) \rightarrow Q(x))$\\
        Q(Eight Men Out)\\
        = P(Eight Men Out)\\
        By seeing P(x) is False and Q(x) is True then the hypothesis would be True because $p(x) \rightarrow Q(x)$ would be True and Q(x) would also be True. But then the Conclusion P(Eight Men Out) Would be False because P is False making the whole statement false.
    \end{description}
\end{enumerate}

\item The domain for variables $x$ and $y$ is a group of people. The predicate $F(x, y)$ is true if and only if $x$ is a friend of $y$. For the purposes of this problem, assume that for any person $x$ and person $y$, either $x$ is a friend of $y$ or $x$ is an enemy of $y$. Therefore, $\neg F(x, y)$ means that $x$ is an enemy of $y$.

Translate each statement into a logical expression. Then negate the expression by adding a negation operation to the beginning of the expression. Apply De Morgan's law until the negation operation applies directly to the predicate and then translate the logical expression back into English.

\begin{enumerate}
    \item Everyone is a friend of everyone.
    \begin{description}
        \item[Answer:] .\\
        Logical Statement: $\forall x, \forall y(F(x,y))$\\
        Negation: $\lnot \forall x, \forall y(F(x,y))$\\
        DeMorgan's Law: $\exists x, \exists y(\lnot F(x,y))$\\
        English: There is a person who is enemies with someone.
    \end{description}
    \item Everyone is their own friend.
    \begin{description}
        \item[Answer:] .\\
        Logical Statement: $\forall x, \forall y(F(x,y) \land x=y)$\\
        Negation: $\lnot \forall x, \forall y(F(x,y) \land x=y)$\\
        DeMorgan's Law: $\exists x, \exists y(\lnot F(x,y) \lor (x \neq y))$\\
        English: There is a person who is enemies with someone or is not the same person as that someone.
    \end{description}
    \item At least two different people are friends.
    \begin{description}
        \item[Answer:] .\\
        Logical Statement: $\exists x, \exists y (F(x,y) \land (x \neq y))$\\
        Negation: $\lnot \exists x, \exists y (F(x,y) \land (x \neq y))$\\
        DeMorgan's Law: $\forall x, \forall y (\lnot F(x,y) \lor (x=y))$\\
        English: Everyone is enemies with everyone or they are themselves
    \end{description}
    \item Everyone is a friend of someone.
    \begin{description}
        \item[Answer:] .\\
        Logical Statement: $\forall x, \exists y(F(x,y))$\\
        Negation: $\lnot \forall x, \exists y(F(x,y))$\\
        DeMorgan's Law: $\exists x, \forall y(\lnot F(x,y))$\\
        English: There is a person who is enemies with everyone.
    \end{description}
\end{enumerate}





\item Suppose we have the following assumptions:

\begin{enumerate}
    \item ``Logic is difficult or not many students like logic''
    \item ``If mathematics is easy, then logic is not difficult.''
\end{enumerate}


Translate these assumptions into statements involving propositional variables and logical connectives. Using the same notations, translate each following statement and determine whether they are valid conclusions of the assumptions. Justify your answers.

\begin{enumerate}[label=(\roman*)]
    \item Mathematics is not easy, if many students like logic.
    \item Not many students like logic, if mathematics is not easy.
    \item Mathematics is not easy or logic is difficult.
    \item If not many students like logic, then either mathematics is not easy or logic is not difficult.\\
        
    p: logic is easy\\
    q: many students like logic\\
    r: mathematics is easy
    \begin{description}
        \item[(a)] $\lnot p \lor \lnot q$
        \item[(b)] $r \rightarrow$ 
    \end{description}
    .
    \begin{description}
        \item[(i)] $q \rightarrow \lnot r$\\
        \begin{displaymath}
            \begin{array}{|c c c|c|}
            p & q & r & (((\lnot p \lor \lnot q) \land (r \rightarrow p)) \rightarrow (q \rightarrow \lnot r))\\
            \hline
            F & F & F & T\\
            F & F & T & T\\
            F & T & F & T\\
            F & T & T & T\\
            T & F & F & T\\
            T & F & T & T\\
            T & T & F & T\\
            T & T & T & T\\
            \end{array}
        \end{displaymath}\\
        True, the statement is a tautology.
        \item[(ii)] $\lnot r \rightarrow \lnot q$\\
        \begin{displaymath}
            \begin{array}{|c c c|c|}
            p & q & r & (((\lnot p \lor \lnot q) \land (r \rightarrow p)) \rightarrow (\lnot r \rightarrow \lnot q))\\
            \hline
            F & F & F & T\\
            F & F & T & T\\
            F & T & F & F\\
            F & T & T & T\\
            T & F & F & T\\
            T & F & T & T\\
            T & T & F & T\\
            T & T & T & T\\
            \end{array}
        \end{displaymath}\\
        False, the statement is a contradiction at p = F, q = T, r = F.
        \item[(iii)] $\lnot r \lor \lnot p$\\
        \begin{displaymath}
            \begin{array}{|c c c|c|}
            p & q & r & (((\lnot p \lor \lnot q) \land (r \rightarrow p)) \rightarrow (\lnot r \lor \lnot p))\\
            \hline
            F & F & F & T\\
            F & F & T & T\\
            F & T & F & T\\
            F & T & T & T\\
            T & F & F & T\\
            T & F & T & F\\
            T & T & F & T\\
            T & T & T & T\\
            \end{array}
        \end{displaymath}\\
        False, the statement is a contradiction at p = T, q = F, r = T.
        \item[(iv)] $\lnot q \rightarrow (\lnot r \lor p)$\\
        \begin{displaymath}
            \begin{array}{|c c c|c|}
            p & q & r & (((\lnot p \lor \lnot q) \land (r \rightarrow p)) \rightarrow (\lnot q \rightarrow (\lnot r \lor p)))\\
            \hline
            F & F & F & T\\
            F & F & T & T\\
            F & T & F & T\\
            F & T & T & T\\
            T & F & F & T\\
            T & F & T & T\\
            T & T & F & T\\
            T & T & T & T\\
            \end{array}
        \end{displaymath}\\
        True, the statement is a tautology.
    \end{description}
\end{enumerate}
\end{enumerate}
\end{document}