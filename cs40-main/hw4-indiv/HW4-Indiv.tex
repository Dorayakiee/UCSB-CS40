\documentclass[12pt, oneside]{article}



\documentclass{article}
\usepackage[letterpaper, scale=0.89, centering]{geometry}
\usepackage{amssymb,amsmath,pifont,amsfonts,comment,enumerate}
\usepackage{currfile,xstring,hyperref,tabularx,graphicx,wasysym}
\usepackage[labelformat=empty]{caption}
\usepackage[dvipsnames,table]{xcolor}
\usepackage{multicol,multirow,array,listings,tabularx,lastpage,textcomp,booktabs}



% NOTE: This environment is credit @pnpo (https://tex.stackexchange.com/a/218450)
\lstnewenvironment{algorithm}[1][] %defines the algorithm listing environment
{   
    \lstset{ %this is the stype
        mathescape=true,
        frame=tB,
        numbers=left, 
        numberstyle=\tiny,
        basicstyle=\rmfamily\scriptsize, 
        keywordstyle=\color{black}\bfseries,
        keywords={,procedure, div, for, to, input, output, return, datatype, function, in, if, else, foreach, while, begin, end, }
        numbers=left,
        xleftmargin=.04\textwidth,
        #1
    }
}
{}
\lstnewenvironment{java}[1][]
{   
    \lstset{
        language=java,
        mathescape=true,
        frame=tB,
        numbers=left, 
        numberstyle=\tiny,
        basicstyle=\ttfamily\scriptsize, 
        keywordstyle=\color{black}\bfseries,
        keywords={, int, double, for, return, if, else, while, }
        numbers=left,
        xleftmargin=.04\textwidth,
        #1
    }
}
{}

\newcommand\abs[1]{\lvert~#1~\rvert}
\newcommand{\st}{\mid}

\newcommand{\A}[0]{\texttt{A}}
\newcommand{\C}[0]{\texttt{C}}
\newcommand{\G}[0]{\texttt{G}}
\newcommand{\U}[0]{\texttt{U}}

\newcommand{\cmark}{\ding{51}}
\newcommand{\xmark}{\ding{55}}

\setlength{\parindent}{0em}
\setlength{\parskip}{1em}

\hypersetup{
    colorlinks=true,
    linkcolor=blue,
    filecolor=magenta,      
    urlcolor=cyan,
    pdftitle={Sharelatex Example},
    bookmarks=true,
    pdfpagemode=FullScreen,
}
\usepackage{graphicx} % Required for inserting images

\title{HW1}
\author{Zixuan Chen}
\date{June 2024}

\begin{document}

\maketitle

\section{Introduction}

\end{document}


\usepackage{enumitem}

\title{HW4 Individual}
\author{CS40 Spring '22\\\\
Bharat Kathi (5938444)}
\date{Due: Friday, May 6, 2022 at 11:59PM on Gradescope}
\begin{document}
\maketitle

{\bf Integrity reminders for individual homeworks}
\begin{itemize}
\item ``Individual homeworks'' must be solely your own work. 
\item You may not collaborate on individual homeworks with anyone or seek help from online tutors or entities outside the class.
\item You may ask questions about the homework in office hours (of the instructor, TAs, and/or tutors) and 
on Piazza.  However, the staff will only answer clarifying questions on these homeworks. You \emph{cannot} use any online resources about the course content other than the text
book and class material from this quarter.
\item Do not share written solutions or partial solutions for homework with other students. Doing so would dilute their learning experience and detract from their success in the class.
\end{itemize}

You will submit this assignment via Gradescope
(\href{https://www.gradescope.com}{https://www.gradescope.com}) in the assignment called ``HW4-Individual''.

\subsection*{Summary of Proof Strategies (so far)}
In your proofs and disproofs of statements below, justify each  step
by reference to  a component of the  following proof  strategies
we  have discussed so far, and/or to relevant definitions and calculations.
\begin{itemize}
    \item A counterexample can be used to prove that  $\forall x P(x)$ is {\bf false}.
    \item  A witness can be used  to  prove that  $\exists x P(x)$ is {\bf true}.
    \item {\bf Proof of universal by exhaustion}: To prove that $\forall x \, P(x)$
is true when $P$ has a finite domain, evaluate the predicate at {\bf each} domain element to confirm that it is always T.
    \item  {\bf Proof by universal generalization}: To prove that $\forall x \, P(x)$
is true, we can take an arbitrary element $e$ from the domain and show that $P(e)$ is true, without making any assumptions about $e$ other than that it comes from the domain.
    \item To  prove  that $\exists x P(x)$ is {\bf false}, write the universal statement that is logically equivalent to its negation and then prove it true using universal generalization.
    \item {\bf Strategies for conjunction}: To prove that $p \land q$ is true, have two subgoals: subgoal (1) prove $p$ 
is  true; and, subgoal (2) prove $q$ is true. To prove that $p \land q$ is false, it's enough to prove that $p$ is false.
 To prove that $p \land q$ is false, it's enough to prove that $q$ is false.
    \item {\bf Proof of Conditional by Direct Proof}: To prove that the implication $p \to q$ is true, we can assume $p$ is true and use that assumption to show $q$ is true.
    \item {\bf Proof of Conditional by Contrapositive Proof}: To prove that the implication $p \to q$ is true, we can assume $\neg q$ is true and use that assumption to show $\neg p$ is true.
   
\end{itemize}


\newpage
\section*{Assigned Questions}
\begin{enumerate}




% \item Consider the predicate  $F(a,b)  = ``a \text{ is a factor of } b"$ over  the domain $\mathbb{Z}^{\neq 0} \times \mathbb{Z}$ that was introduced in lecture. Consider the following quantified
% statements
% \label{factoring}

% \begin{multicols}{2}
% \begin{enumerate}[label=(\roman*)]
% \item $\forall x \in \mathbb{Z} ~(F(1,x))$
% \item $\forall x \in \mathbb{Z}^{\neq 0} ~(F(x,1))$
% \item $\exists x \in \mathbb{Z} ~(F(1,x))$
% \item $\forall x \in \mathbb{Z}^{\ne 0} ( \neg F(x,1))$
% \item $\forall x \in \mathbb{Z}^{\neq 0} ~\exists y \in \mathbb{Z} ~(F(x,y))$
% \item $\exists x \in \mathbb{Z}^{\neq 0} ~\forall y \in \mathbb{Z} ~(F(x,y))$
% \item $\forall y \in \mathbb{Z} ~\exists x \in \mathbb{Z}^{\neq 0} ~(F(x,y))$
% \item $\exists y \in \mathbb{Z} ~\forall x \in \mathbb{Z}^{\neq 0} ~(F(x,y))$
% \end{enumerate}
% \end{multicols}

% \begin{enumerate}

% \item ({\it Graded for correctness of choice and fair effort completeness in justification}) 
% Which of the statements (i) - (viii) is being {\bf proved} by the following proof?

% \begin{quote}
%   By universal generalization, {\bf choose} $e$ to be an {\bf arbitrary} non-zero integer. 
%   We need to show $\exists b \in \mathbb{Z} (F(e,b))$. By definition of the  predicate $F$, we can rewrite 
%   this goal as $\exists b \in \mathbb{Z}  \exists c \in \mathbb{Z}~(b = c \cdot e)$. We pick the {\bf witnesses} $b = e$  and $c = 1$. $b$ is an integer and therefore in the domain. Similarly, $c$ is an integer and therefore in the domain. Plugging the value of the witnesses $b$ and $c$, we get 
%   $b = c \cdot e \to e = 1 \cdot e \to e = e$, as required. Since the predicate $\exists b \in \mathbb{Z} (F(e,b))$ evaluates to true 
%   for the arbitrary, non-zero, integer $e$, the claim has been proved.
  
%   \hfill $\blacksquare$
% \end{quote}


% {\it Hint: It may be useful to 
% identify the key words in the proof that indicate proof strategies.}

% \item ({\it Graded for correctness of choice and fair effort completeness in justification}) 
% Which of the statements (i) - (viii) is being {\bf disproved} by the following proof? 

% \begin{quote}
%   To disprove the statement, we need to find a counterexample. We choose $2$, which is a nonzero
%   integer so in the domain. We need to show $ \neg F(2,1)$. By definition of the predicate $F$, we 
%   can rewrite this goal as $\neg \exists c \in \mathbb{Z}~(1 = c \cdot 2)$. Evaluating this statement, $1 = c \cdot 2 \to c = 1/2$. Since $1/2$ is not an integer, $c$ is not in the domain. Therefore, the predicate $F(2,1)$ is false. So the counterexample works to 
%   disprove the original statement.
%   \hfill $\blacksquare$
% \end{quote}

% {\it Hint: It may be useful to 
% identify the key words in the proof that indicate proof strategies.}



% \end{enumerate}



\item \textbf{Theorem}: If $n$ and $m$ are odd integers, then $n \cdot m$ is odd. 

For each of the following proof attempts of the theorem, explain where the proof uses invalid reasoning or skips essential steps.
\begin{enumerate}
    \item Let $n$ and $m$ be odd integers. Then $n=2k+1$ and $m=2j+1$. Plugging into the expression $n\cdot m$ gives
    \[n \cdot m = (2k+1)(2j+1)=2(j\cdot k+j+k)+1\]
    Since $k$ and $j$ are integers, $j\cdot k+j+k$ is also an integer. Then $n\cdot m$ equals to two times an integer plus one, therefore $n\cdot m$ is odd.
    
    \begin{description}
        \item[Answer:] The proof should state that the definition of an odd number is being used to show that n = 2k + 1 and m = 2j + 1. The proof should also state that k and j are some arbitrary integers. The steps in multiplying (2k+1) and (2j+1) need to be shown. The result is missing a coeffcient of 2 in front of $j * k$.
    \end{description}
    
    \item Let $n$ and $m$ be odd integers. Since $n$ is an odd integer, then $n = 2k+1$ for some integer $k$. Since $m$ is an odd integer, then $m = 2j+1$ for some integer $j$. Plugging in $2k+1$ for $n$ and $2j+1$ for $m$ into the expression $n\cdot m$ gives
    \[n\cdot m=(2k+1)(2j+1)\]
    Since $n\cdot m$ is equal to two times an integer plus one, then $n\cdot m$ is an odd integer.
    
    \begin{description}
        \item[Answer:] The proof needs to show the steps in multiplying (2k+1) and (2j+1).
    \end{description}
    
    \item Let $n$ and $m$ be odd integers. Since $n$ is an odd integer, then $n = 2k+1$ for some integer $k$. Since $m$ is an odd integer, then $m = 2k+1$ for some integer $k$. Plugging into the expression $n\cdot m$ gives
    \[n\cdot m=(2k+1)(2k+1)=2(2k^2+2k)+1\]
    Since $k$ is an integer, $2k^2+2k+1$ is also an integer. Since $n\cdot m$ is equal to two times an integer plus one, then $n\cdot m$ is an odd integer.

    \begin{description}
        \item[Answer:] Since m and n are two distinct integers, they need sepereate defintions for being odd numbers. They cannot both use the variable k. In the current setup, m and n are both equal to 2k+1 which would mean that m and n are equal.
    \end{description}

\end{enumerate}

\quad
% \item \textbf{Theorem}: If $w, x, y, z$ are non-zero integers where $w$ divides $x$ and $y$ divides $z$, then $wy$ divides $xz$.

% For each proof attempt of the theorem, explain where the proof uses invalid reasoning or skips essential steps.
% \begin{enumerate}
%     \item By assumption, $w$ divides $x$, so $x = kw$ for some integer $k$ and $w \neq 0$. Similarly, $z = ky$ for some integer $k$ and $y \neq 0$. Plug in the expression $kw$ for $x$ and $ky$ for $z$ in the expression $xz$ to get
%     \[xz=(kw)(ky)=(k^2)(wy)\]
%     Since $k$ is an integer, then $k^2$ is also an integer. Since $w \neq 0$ and $y \neq 0$, then $wy \neq 0$. Since $xz$ equals $wy$ times an integer and $wy \neq 0$, then $wy$ divides $xz$.
    
%     \item By assumption, $w$ divides $x$, so $x = kw$ for some integer $k$ and $w \neq 0$. Similarly, $z = jy$ for some integer $j$ and $y \neq 0$. Plug in the expression $kw$ for $x$ and $jy$ for $z$ in the expression $xz$ to get
%     \[xz=(kw)(jy)\]
%     Since $w \neq 0$ and $y \neq 0$, then $wy \neq 0$. Since $xz$ equals $wy$ times an integer and $wy \neq 0$, then $wy$ divides $xz$.
    
%     \item By assumption, $w$ divides $x$, so $x = kw$ and $w \neq 0$. Similarly, $z = jy$ and $y \neq 0$. Plug in the expression $kw$ for $x$ and $jy$ for $z$ in the expression $xz$ to get
%     \[xz=(kw)(jy)=(kj)(wy)\]
%     Since $k$, $j$ are integers, then $kj$ is an integer. Since $w \neq 0$ and $y \neq 0$, then $wy \neq 0$. Since $xz$ equals $wy$ times an integer and $wy \neq 0$, then $wy$ divides $xz$.
% \end{enumerate}


\item \textbf{Theorem} For all non-zero integers, $x, y, z$, if $x$ does not divide $yz$, then $x$ does not divide $y$ 

For each of the following proof attempts of the theorem, explain where the proof uses invalid reasoning or skips essential steps.
\quad
\begin{enumerate}
    \item We prove this by contrapositive. Let $x,y, z$ be non-zero integers. We assume that $x$ divides $y$. Then $y=kx$. So, \[zy= z(kx)=(kz)x\] Since $z, k$ are integers, $zk$ is also an integer. So, $x$ divides $yz$.
    
    \begin{description}
        \item[Answer:] The proof should explicitly state that the definition of being divisible is used, and that k is some arbitrary integer. It also needs to show some steps and that z is being multiplied to both sides of the equation. It should also explicitly state that yz is of the form mx, where m = zk is some integer.
    \end{description}

    \item We prove this by contrapositive. Let $x,y, z$ be non-zero integers. We assume that $x$ divides $y$. So, $y=kx$ for some integer $k$. Since $zk$ is also an integer, we conclude that $x$ divides $zy$, thus proving the theorem.
    
    \begin{description}
        \item[Answer:] The proof is missing all the steps to show that yz is of the form mx, where m = zk is an integer and x is nonzero.
    \end{description}

    \item We prove this by contrapositive. Let $x,y, z$ be non-zero integers. Assume that $x$ divides $y$. So, $x=ky$ for some integer $k$. We further assume that $ky=ayz$ for some integer $a$, and show that $x$ divides $yz$. We divide  $ky=ayz$ out to get $a=\frac{k}{z}$. Since $k, z$ are both integers, $kz$ is also an integer, thus proving that $x$ divides $yz$.

    \begin{description}
        \item[Answer:] The proof has an incorrect definition of x | y, and also assumes that x | yz.
    \end{description}

\end{enumerate}

\item Consider the statement: The sum of any two integers is odd if and only if at least one of them is odd.
\begin{enumerate}
    \item Define predicates as necessary and write the symbolic form of the statement using quantifiers.
    
    \begin{description}
        \item[Answer:] .\\
        Let O(n) be the predicate "n is odd". \\
        $\forall x \in \mathbb{Z}, \forall y \in \mathbb{Z} (O(x+y) \leftrightarrow (O(x) \lor O(y)))$
    \end{description}
    
    \item Prove or disprove the statement. Specify which proof strategy is used.
    
    \begin{description}
        \item[Answer:] .\\
        Disproving by counterexample. \\
        Let x = 1 and y = 1. \\
        The proposition is now $(O(1+1) \leftrightarrow (O(1) \lor O(1)))$\\
        Since $1+1=2$ and 2 is not even, the left side of our conditional is false, while the right side is true.\\
        Our conditional is equivalent to $F \leftrightarrow T$ which evaluates to $F$.\\
        Therefore, the statement is disproven.
    \end{description}

\end{enumerate}

\quad

\item Consider the statement: If $x$ and $y$ are integers such that $x + y \geq 5$, then $x > 2$ or $y > 2$.
\begin{enumerate}
    \item Write the symbolic form of the statement using quantifiers.
    
    \begin{description}
        \item[Answer:] $\forall x \in \mathbb{Z} , \forall y \in \mathbb{Z} ((x+y \geq 5) \rightarrow (x > 2 \lor y>2))$
    \end{description}
    
    \item Prove or disprove the statement. Specify which proof strategy is used.
    
    \begin{description}
        \item[Answer:] .\\
        Proof by contrapositive. \\
        Let x and y be arbitrary integers. \\
        Assume that $x \leq 2 \land y \leq 2$. \\
        We will prove that $x+y < 5$. \\
        Since $x \leq 2$ and $ y \leq 2$, adding the two results in $x+y \leq 4$.
        Since $4<5$, we can say that $x+y<5$. \\
        Therefore, our original statement is proved by the contrapositive.
    \end{description}

\end{enumerate}

\quad

\item Consider the statement: The average of two odd integers is an integer.
\begin{enumerate}
    \item Write the symbolic form of the statement using quantifiers.
    
    \begin{description}
        \item[Answer:] $\forall x \in \mathbb{Z} , \forall y \in \mathbb{Z} ( ( (\exists k \in \mathbb{Z} (x=2k+1)) \land (\exists j \in \mathbb{Z} (y=2j+1) )) \rightarrow  (\frac{x+y}{2} \in \mathbb{Z} ))$
    \end{description}

    \item Prove or disprove the statement. Specify which proof strategy is used.

    \begin{description}
        \item[Answer:] .\\
        Proof by direct proof.\\
        Let x and y be arbitrary integers.\\
        Assume that x and y are odd. We will prove that $\frac{x+y}{2}$ is an integer.\\
        Since x is odd, that means that $x=2k+1$ for some intejer k. \\
        Since y is odd, that means that $y=2j+1$ for some intejer j. \\
        Substitute $x=2k+1$ and $y=2j+1$ into $\frac{x+y}{2}$. \\
        $= \frac{(2k+1)+(2j+1)}{2}$ \\
        $= \frac{(2k+2j+2)}{2}$ \\
        $= \frac{2(k+j+1)}{2}$ \\
        $= k+j+1$ \\
        Since we know that k and j are integers, $k+j+1$ must also be an integer. \\
        Therefore, $\frac{x+y}{2}$ is an integer.
    \end{description}

\end{enumerate}

\quad

\item Consider the statement: For any three consecutive integers, their product is divisible by 6.
\begin{enumerate}
    \item Write the symbolic form of the statement using quantifiers.
    \begin{description}
        \item[Answer:] $\forall x \in \mathbb{Z} , \forall y \in \mathbb{Z} , \forall z \in \mathbb{Z} ( ((y=x+1) \land (z = x+2)) \rightarrow (6 | xyz) )$
    \end{description}
    \item Prove or disprove the statement. Specify which proof strategy is used.
    \begin{description}
        \item[Answer:] .\\
        Proof by cases. \\
        Since x, y, and z are all consecutive, we can write them as the following: \\
        $x = x$ \\
        $y = x+1$ \\
        $z = x+3$ \\
        Our 3 consecutive numbers are now x, x+1, and x+2. \\
        So $xyz = x(x+1)(x+2)$ \\
        Since 6 = 2 x 3, an integer that is divisble by both 2 and 3 must also be divisible by 6. \\
        Therefore, we can prove $6|x(x+1)(x+2)$ is divisble by 6 by proving $(2|x(x+1)(x+2) \land 3|x(x+1)(x+2))$. \\
        
        Lets look at $2|x(x+1)(x+2)$ first.\\
        
        Case 1: x is even. \\
        This means that $x=2k$, for some integer k since x is even. \\
        Substitute $x=2k$ into $x(x+1)(x+2)$. \\
        $x(x+1)(x+2) = 2k(2k+1)(2k+2)$ \\
        $x(x+1)(x+2)$ can be written as $2m$ where $m = k(2k+1)(2k+2)$. \\
        Since k is an integer, $k(2k+1)(2k+2)$ is also an integer, making $m$ an integer as well. \\
        Therefore, $x(x+1)(x+2)$ is divisble by 2. \\
        
        Case 2: x is odd. \\
        This means that $x=2k+1$, for some integer k since x is odd. \\
        Substitute $x=2k+1$ into $x(x+1)(x+2)$. \\
        $x(x+1)(x+2) = (2k+1)(2k+2)(2k+3)$ \\
        $= (2k+1)(2(k+1))(2k+3)$ \\
        $= 2(2k+1)(k+1)(2k+3)$ \\
        $x(x+1)(x+2)$ can be written as $2m$ where $m = (2k+1)(k+1)(2k+3)$. \\
        Since k is an integer, $(2k+1)(k+1)(2k+3)$ is also an integer, making $m$ an integer as well. \\
        Therefore, $x(x+1)(x+2)$ is divisble by 2. \\

        Now that we have proven $2|x(x+1)(x+2)$, lets look at $3|x(x+1)(x+2)$. \\

        Case 1: x is a multiple of 3. \\
        This means that $x=3k$ for some integer k. \\
        Substitute $x=3k$ into $x(x+1)(x+2)$. \\
        $x(x+1)(x+2) = 3k(3k+1)(3k+2)$ \\
        $x(x+1)(x+2)$ can be written as $3m$ where $m = k(3k+1)(3k+2)$. \\
        Since k is an integer, $k(3k+1)(3k+2)$ is also an integer, making $m$ an integer as well. \\
        Therefore, $x(x+1)(x+2)$ is divisble by 3. \\

        Case 2: x is one number greater than a multiple of 3. \\
        This means that $x=3k+1$ for some integer k. \\
        Substitute $x=3k+1$ into $x(x+1)(x+2)$. \\
        $x(x+1)(x+2) = (3k+1)(3k+2)(3k+3)$ \\
        $= (3k+1)(3k+2)(3(k+1))$ \\
        $= 3(3k+1)(3k+2)(k+1)$ \\
        $x(x+1)(x+2)$ can be written as $3m$ where $m = (3k+1)(3k+2)(k+1)$. \\
        Since k is an integer, $(3k+1)(3k+2)(k+1)$ is also an integer, making $m$ an integer as well. \\
        Therefore, $x(x+1)(x+2)$ is divisble by 3. \\

        Case 3: x is two numbers greater than a multiple of 3. \\
        This means that $x=3k+2$ for some integer k. \\
        Substitute $x=3k+2$ into $x(x+1)(x+2)$. \\
        $x(x+1)(x+2) = (3k+2)(3k+3)(3k+4)$ \\
        $= (3k+1)(3(k+1))(3k+4)$ \\
        $= 3(3k+1)(k+1)(3k+4)$ \\
        $x(x+1)(x+2)$ can be written as $3m$ where $m = (3k+1)(k+1)(3k+4)$. \\
        Since k is an integer, $(3k+1)(k+1)(3k+4)$ is also an integer, making $m$ an integer as well. \\
        Therefore, $x(x+1)(x+2)$ is divisble by 3. \\

        Now that we have proven both $2|x(x+1)(x+2)$ and $3|x(x+1)(x+2)$, we have proven that $6|x(x+1)(x+2)$.\\
        Therefore, we have proven that $6|xyz$.
    \end{description}
\end{enumerate}

\quad

\item Consider the following statements:
\label{quant_statements}
\begin{enumerate}[label=(\roman*)]
    \item If $x$ and $y$ are even integers, then $x+y$ is an even integer.
    \item If $x+y$ is an even integer, then and $x$ and $y$ are both even integers.
    \item If $x$ and $y$ are integers and $x^2 = y^2$, then $x=y$.
    \item If $x$ and $y$ are real numbers and $x<y$, then $x^2 < y^2$.
    \item If $x$ and $y$ are positive real numbers and  $x<y$, then $x^2 < y^2$.
\end{enumerate}

\begin{enumerate}
    \item ({\it Graded for correctness\footnote{This means your solution will be
evaluated not only on the correctness of your answers, but on your ability to 
present your ideas clearly and logically. You should explain how you arrived at your conclusions, using 
mathematically sound reasoning. Whether you use formal proof techniques or write a more informal argument for why 
something is true, your answers should always be well-supported. Your goal should be to convince the reader that 
your results and methods are sound.}}) Express the statements (i) - (v) as quantified statements. Define any predicates as needed.

\begin{description}
    \item[Answer:] .\\
    Let F(x) be defined as "x is even" \\\\
    (i) $\forall x \in \mathbb{Z}, \forall y \in \mathbb{Z} ((F(x) \land F(y)) \rightarrow F(x+y))$ \\
    (ii) $\forall x \in \mathbb{Z}, \forall y \in \mathbb{Z} (F(x+y) \rightarrow (F(x) \land F(y)))$ \\
    (iii) $\forall x \in \mathbb{Z}, \forall y \in \mathbb{Z} ((x^2=y^2) \rightarrow (x=y))$ \\
    (iv) $\forall x \in \mathbb{R}, \forall y \in \mathbb{R} ((x<y) \rightarrow (x^2 < y^2))$ \\
    (v) $\forall x \in \mathbb{R^+}, \forall y \in \mathbb{R^+} ((x<y) \rightarrow (x^2<y^2))$
\end{description}

\item ({\it Graded for correctness of choice and fair effort completeness in justification}\footnote{This means that for the justification, you will get full credit so long as your submission 
demonstrates honest effort to answer the question. You will not be penalized for an incorrect justification.}) What are the error(s) in the following attempted proof for statement (ii)?

\begin{quote}
  We prove this by contrapositive. So, by \textbf{universal generalization} let $x$ and $y$ be arbitrary integers, such that $x$ and $y$ are not both even. Without loss of generality, suppose that $x$ is even and $y$ is odd. Then $x=2k$ for some integer $k$ and $y=2j+1$ for some integer $j$. $x+y=2k+2j+1=2(k+j)+1$. Since $j$ and $k$ are both integers, $j+k$ is too. Therefore, $x+y$ is odd. This proves the contrapositive, thus proving the claim. \hfill{$\blacksquare$}
\end{quote}

\begin{description}
    \item[Answer:] Universal generalization can only be used with the cases "x is even and y is odd" and "x is odd and y is even" wihtout loss of generality. The proof for these two cases does not work for the case "both x and y are odd". So a seperate proof must be used for this case.
\end{description}

\item Write the correct version of the proof for statement (ii) if it is true or provide a counterexample to disprove it if it is not.

\begin{description}
    \item[Answer:] Counterexample: x = 5, y = 5 \\
    The sum $x+y=10$ which is even.\\
    However, both $x$ and $y$ are odd (not even), rendering statement (ii) false.
\end{description}

\item ({\it Graded for correctness of choice and fair effort completeness in justification}) 
Which of the statements (i) - (v) is being {\bf disproved} by the following proof?

\begin{quote}
  To disprove the statement, we need to find a counterexample. We need to show that there exist some $x$ and $y$ in the domain such that
  $\neg ~(~(x^2=y^2) \to (x=y)~)$. Rewriting this goal using the disjunctive form of the implication and applying De Morgan's Law, we need to show  $~(~(x^2=y^2) \land \neg(x=y)~)$.
  
  We choose the witnesses $x = -1$ and $y=1$, which are both in the domain and satisfy the condition $x^2 =y^2$. However, since $-1\neq 1$, the counterexample works to disprove the original statement.
  \hfill{$\blacksquare$}
\end{quote}

\begin{description}
    \item[Answer:] Statement (iii) is being disproved by this proof.\\
    Since they are trying to find a counterexample using $\neg ~(~(x^2=y^2) \to (x=y)~)$, that means they are trying to disprove the statement $(~(x^2=y^2) \to (x=y)~)$, which is statement (iii).
\end{description}

\end{enumerate}
\item
We define the set of balanced parentheses $S$ recursively as follows:

Basis: the string $()$ is in $S$

Recursive rules:

\begin{enumerate}
    \item $\forall x \in S( (x) \in S)$
    \item $\forall x \in S \forall y \in S (xy \in S)$ where $xy$ means the concatenation of $x$ and $y$
\end{enumerate}

Using structural induction, prove that for any string $a$ in $S$, the number of left parentheses in $x$ is equal to the number of right parentheses in $x$.

\begin{description}
    \item[Answer:] .\\
    Let $L[x]$ return the number of left parentheses in $x$. \\
    Let $R[x]$ return the number of right parentheses in $x$.\\
    
    Theorem: $\forall s \in S (L[s] = R[s])$ \\

    Proof by Induction. \\
    Base Case: s = (). L[()] = R[()] = 1. \\
    Inductive Step: Assuming that $x \in S$ and $y \in S$, then L[x] = R[x] and L[y] = R[y]. Then we consider the following 2 cases depending on which was the last recursive rule.\\

    Case 1: $s=(x)$, where $x \in S$. We assume that L[x] = R[x] and prove that L[s] = R[s].\\
    
    $L[s] = L[(x)]$ , Since $s=(x)$.\\
    $= 1+L[x]$ , Since (x) has one more ( than x.\\
    $= 1+R[x]$ , By the inductive hypothesis.\\
    $= R[(x)]$ , Since (x) has one more ) than x.\\
    $= R[s]$ , Since $s=(x)$.\\

    Case 2: $s=xy$, where $x \in S$ and $y \in S$. We assume the inductive hypothesis and prove that L[s] = R[s].\\

    $L[s] = L[xy]$, Since $s=xy$.\\
    $=L[x] + L[y]$ \\
    $=R[x]+R[y]$, By the inductive hypothesis.\\
    $=R[xy]$ \\
    $=R[s]$, Since $s=xy$. \\

    Therefore, L[s] = R[s].

\end{description}

\end{enumerate}



\end{document}
